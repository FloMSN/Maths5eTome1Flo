\serie{Comparaison}


\begin{exercice}[Signes]
Donne le signe des nombres suivants :

$\dfrac{-5,2}{4,23}$ ; $\dfrac{5}{-2,1}$ ; $\dfrac{472}{23}$ ; $\dfrac{-8,9}{-45}$ ; $-\dfrac{12}{13}$ ; $-\dfrac{11}{-5,2}$.
\end{exercice}





\begin{exercice}
Indique les nombres égaux parmi ceux de la liste ci-dessous :

$\dfrac{-8}{9}$ ; $-\dfrac{8}{9}$ ; $\dfrac{-8}{-9}$ ; $-\dfrac{8}{-9}$ ; $\dfrac{8}{-9}$ ; $-\dfrac{-8}{9}$ ; $\dfrac{8}{9}$.
\end{exercice}




\begin{exercice}[Encadrement]

\begin{enumerate}
\item On considère la fraction $\dfrac{56}{21}$.

Effectue la division euclidienne de 56 par 21 et déduis-en un encadrement de la fraction par deux nombres entiers consécutifs.
\item Encadre $\dfrac{-89}{15}$ puis $\dfrac{47}{59}$ par deux nombres entiers consécutifs.
\item Encadre respectivement $\dfrac{-47}{25}$ et $\dfrac{13}{-4}$ par deux nombres entiers consécutifs et déduis-en la comparaison de ces deux fractions.
\item Peux-tu appliquer la même méthode pour comparer $\dfrac{25}{3}$ et $\dfrac{90}{11}$ ?
\end{enumerate}
\end{exercice}




\begin{exercice}[Avec des valeurs approchées]
Soient deux nombres : $a =\dfrac{816}{577}$  et $b =\dfrac{577}{408}$.
\begin{enumerate}
\item Donne la valeur arrondie de $a$ et celle de $b$ au millième. Peux-tu en déduire la comparaison de $a$ et de $b$ ?
\item Donne des valeurs approchées de $a$ et $b$ qui permettent de les comparer. Compare $a$ et $b$.
\end{enumerate}
\end{exercice}





\begin{exercice}[Égalités]
Recopie et complète chacune des égalités suivantes :
\begin{enumerate}
\item $\dfrac{...}{-5}=\dfrac{10}{20}$
\item $\dfrac{2}{3}=\dfrac{...}{27}$
\item $\dfrac{-15}{45}=\dfrac{-5}{...}$
\item $\dfrac{...}{-18}=\dfrac{7}{6}$
\item $3=\dfrac{...}{4}$
\item $-2,1=-\dfrac{21}{...}$
\end{enumerate}
\end{exercice}




\begin{exercice}
Dans chaque cas, à partir des égalités données et en utilisant seulement les quatre nombres qui apparaissent, écris toutes les égalités d'écritures fractionnaires possibles :
\begin{enumerate}
\item $7 \times (-8) = -4 \times 14$
\item $-3 \times (-1) = 2 \times 1,5$
\item $2,1 \times 12 = 9 \times 2,8$
\item $-4 \times 9 = 12 \times (-3)$
\end{enumerate}
\end{exercice}




\begin{exercice}[Égalité ?]
Recopie et complète en utilisant $=$ ou $\neq$, en justifiant dans chaque cas :
\begin{enumerate}
\item $\dfrac{9}{5} ... \dfrac{26}{15}$ 
\item $\dfrac{-7}{-3} ... \dfrac{-14}{6}$
\item $\dfrac{-12,7}{-5} ... \dfrac{25,4}{10}$
\item $\dfrac{-27,35}{27,35} ... \dfrac{15,72}{-15,72}$
\end{enumerate}
\end{exercice}





\begin{exercice}[Avec un dénominateur entier positif]
Réécris chacune des écritures fractionnaires suivantes avec un dénominateur entier positif :
$\dfrac{4}{-5}$ ; $\dfrac{-8}{-7}$ ; $-\dfrac{5,2}{-7}$ ; $\dfrac{7}{-2,1}$ ; $\dfrac{8,2}{0,12}$ ; $-\dfrac{-1}{-3,54}$.
\end{exercice}





\begin{exercice}[Même dénominateur positif]
\begin{enumerate}
\item Recopie et complète la phrase suivante :

« Deux nombres en écriture fractionnaire de même dénominateur positif sont rangés... ».
\item Compare les nombres suivants :

$\dfrac{-7,5}{3}$ et $\dfrac{-7,49}{3}$ ;

$\dfrac{4,05}{2,1}$ et $\dfrac{4,2}{2,1}$ ;

$-\dfrac{0,74}{5}$ et $\dfrac{-0,7309}{5}$ ; 

$\dfrac{8}{-5,23}$ et $\dfrac{-7,9}{5,23}$. 
\end{enumerate}
\end{exercice}





\begin{exercice}[Avec le même numérateur]
\begin{enumerate}
\item Recopie et complète la phrase suivante :

« Deux nombres positifs en écriture fractionnaire de même numérateur sont rangés… »
\item Compare les nombres suivants :

$\dfrac{3,5}{8,2}$ et $\dfrac{3,5}{8,15}$ ;

$-\dfrac{-1}{6}$ et $\dfrac{1}{5,7}$.
\end{enumerate}
\end{exercice}




\begin{exercice}[Avec le même numérateur (bis)]
Compare les nombres suivants en commençant par comparer leurs opposés :
\begin{enumerate}
\item $\dfrac{1}{-5}$ et $\dfrac{1}{-7}$ ;
\item $\dfrac{-3}{8}$ et $\dfrac{-3}{8,2}$ ;
\item $-\dfrac{5,23}{14,5}$ et $\dfrac{-5,23}{14,6}$ ;
\item $\dfrac{-7,5}{0,23}$ et $\dfrac{75}{-2,4}$. 
\end{enumerate}
\end{exercice}





\begin{exercice}
Dans chaque cas, réécris les nombres avec le même dénominateur positif puis compare-les :
\begin{enumerate} 
\item $\dfrac{-5}{4}$ et $\dfrac{-9}{8}$ ;
\item $\dfrac{2,7}{-9}$ et $\dfrac{-1}{3}$ ;
\item -3 et $-\dfrac{20,9}{-7}$ ;
\item $-\dfrac{2}{11}$ et $\dfrac{-5}{33}$ ; 
\item $\dfrac{7}{2,5}$ et $\dfrac{-20,5}{7,5}$ ; 
\item $\dfrac{13}{-27}$ et $\dfrac{-79}{162}$.
\end{enumerate}
\end{exercice}




\begin{exercice}[Multiple commun]
\begin{enumerate}
\item Quels sont les dix premiers multiples de 12 ? Ceux de 18 ? Déduis-en le plus petit multiple non nul commun à 12 et 18, puis un dénominateur commun positif des fractions : $\dfrac{-7}{12}$ et $\dfrac{-11}{18}$.

Compare alors ces deux nombres.
\item La méthode précédente permet-elle de trouver rapidement un dénominateur commun aux nombres : $\dfrac{8}{11}$ et $\dfrac{10}{13}$ ?

Comment en trouver un alors rapidement ? Compare ces deux nombres.
\end{enumerate}
\end{exercice}





\begin{exercice}Dans chaque cas, réécris les nombres avec le même dénominateur positif, puis compare-les :
\begin{enumerate}
\item $\dfrac{-5}{8}$ et $\dfrac{-3,8}{6}$ ;
\item $\dfrac{14}{5}$ et $\dfrac{20}{7}$ ;
\item $\dfrac{3}{-50}$ et $-\dfrac{4}{75}$ ;
\item $\dfrac{54,5}{0,27}$ et $\dfrac{-2,62}{0,13}$.
\end{enumerate}
\end{exercice}






\begin{exercice}
Compare en justifiant :
\begin{enumerate}
\item $-\dfrac{12}{18}$ et $\dfrac{399}{-300}$ ; 
\item $\dfrac{2}{57}$ et $\dfrac{1}{28,4}$ ;
\item $\dfrac{-75}{47}$ et $\dfrac{25}{-15}$ ;
\item $\dfrac{-5}{6}$ et $-\dfrac{15}{14}$ ;
\item $\dfrac{6}{13}$ et $\dfrac{29}{65}$ ;
\item $\dfrac{3}{-22}$ et $\dfrac{4,5}{33}$.
\end{enumerate}
\end{exercice}




\begin{exercice}[Dans l'ordre]
\begin{enumerate}
\item Range les nombres suivants dans l'ordre croissant sans utiliser de valeurs approchées :

$\dfrac{7}{-15}$ ; $\dfrac{7}{3}$ ; $\dfrac{490}{420}$ ; $\dfrac{-5}{12}$ ; $\dfrac{-24}{-18}$ ; 2,5.
\item Range les nombres suivants dans l'ordre décroissant :

$\dfrac{-29}{100}$ ; $\dfrac{7}{-25}$ ; $-0,285$ ; $-\dfrac{1}{5}$ ; $\dfrac{13}{-50}$ ; 0 ; $\dfrac{-1}{2,5}$.
\end{enumerate}
\end{exercice}





\begin{exercice}[Trajet]
Quatre amis font un voyage en trois jours. Le premier jour, ils parcourent 40\,\%\ du trajet total ; le deuxième jour, un quart et le dernier jour, $\dfrac{7}{20}$ du trajet total.

Quel jour ont-ils parcouru la plus grande distance ?

Peux-tu calculer la distance parcourue chaque jour ?
\end{exercice}







\serie{Multiplications}





\begin{exercice}[La règle et les signes]
Effectue les produits suivants :
\begin{enumerate}
\item $\dfrac{3}{2} \times \dfrac{5}{7}$ ;
\item $\dfrac{-4}{11} \times \dfrac{1}{-3}$ ;
\item $3 \times \dfrac{-7}{5}$ ;
\item $\dfrac{5}{-4} \times \dfrac{5}{-2}$ ;
\item $\dfrac{8}{17} \times \dfrac{5}{-3}$ ;
\item $-\dfrac{13}{5} \times \left(-\dfrac{2}{11}\right)$ ;
\item $\left(-\dfrac{7}{15}\right) \times (-8) \times \dfrac{2}{3}$ ;
\item $\dfrac{-1}{2} \times \dfrac{5}{-4} \times \dfrac{-3}{2}$.
\end{enumerate}
\end{exercice}





\begin{exercice}[Toujours plus simple]
Simplifie, si possible, les écritures fractionnaires suivantes :
\begin{enumerate}
\item $\dfrac{-5 \times 2}{2 \times 7}$ ;  
\item $\dfrac{-5 + 2}{7 + 2}$ ;
\item $\dfrac{4 \times (-11)}{4 \times (-11) \times 3}$ ;
\item $\dfrac{8 \times (-3) \times 7 \times 5}{3 \times 5 \times 8 \times 7}$ ;
\item $\dfrac{-5 \times 8}{2 \times (-7)}$ ;
\item $\dfrac{5 \times (-9) \times 2}{(-7) \times 10 \times (-1)}$ ;
\end{enumerate}
\end{exercice}





\begin{exercice}[Calculer en simplifiant]
Pour chacun des produits suivants, applique la règle de multiplication sans effectuer les calculs, simplifie lorsque cela est possible et donne alors le résultat sous la forme d'une fraction irréductible :
\begin{enumerate}
\item $\dfrac{8}{5} \times \dfrac{5}{7}$ ;
\item $\dfrac{-3}{10} \times \dfrac{-11}{3}$ ;
\item $\dfrac{-2}{3} \times \dfrac{-5}{2} \times \dfrac{3}{-7}$ ;
\item $\dfrac{5}{-7} \times \left(-\dfrac{7}{5}\right)$ ;
\item $-15 \times \dfrac{2}{15}$ ;
\item $\left(-\dfrac{8}{3}\right) \times \left(-\dfrac{1}{5}\right) \times 3$.
\end{enumerate}
\end{exercice}




\begin{exercice}Complète les égalités  suivantes :
\begin{enumerate}
\item $\dfrac{8}{...} \times \dfrac{7}{3} = -\dfrac{8}{3}$ ; 
\item $\dfrac{-5}{3} \times \dfrac{7}{...} = \dfrac{7}{6}$ ;
\item $\dfrac{6}{5} \times ... = -6$ ;
\item $\left(-\dfrac{8}{21}\right) \times \dfrac{...}{...} = 1$ ;
\item $\dfrac{...}{10} \times \dfrac{7}{...} = -5$ ;
\item $\dfrac{...}{-9} \times \dfrac{2}{...} = \dfrac{4}{15}$ ;
\item $\dfrac{-5}{...} \times \dfrac{3}{-14} \times \dfrac{...}{25} = \dfrac{-2}{7}$.
\end{enumerate}
\end{exercice}




\begin{exercice}[Simplifier avant de calculer]
\begin{enumerate}
\item Écris 15 sous la forme d'un produit de deux nombres entiers. Décompose de même 20 en produit de nombres entiers positifs les plus petits possibles.
\item Recopie et complète les égalités suivantes :
$\dfrac{15}{7} \times \dfrac{11}{20} = \dfrac{... \times ...}{... \times ...} = \dfrac{(... \times ...) \times ...}{... \times (... \times ... \times ...)}$.
\item Simplifie l'expression obtenue et donne le résultat sous forme d'une fraction irréductible.
\end{enumerate}
\end{exercice}





\begin{exercice}
Calcule les produits suivants en simplifiant, puis donne les résultats sous la forme d'une fraction irréductible :
\begin{enumerate}
\item $\dfrac{-7}{25} \times \dfrac{-5}{8}$ ;
\item $\dfrac{18}{-49} \times \dfrac{14}{27}$ ;
\item $\dfrac{45}{28} \times \dfrac{7}{-15}$ ;
\item $\dfrac{-2}{6} \times \left(-\dfrac{21}{11}\right)$ ;
\item $\dfrac{21}{32} \times \dfrac{108}{49}$ ;
\item $-26 \times \dfrac{-5}{39}$ ;
\item $\dfrac{8}{5} \times \dfrac{-5}{21} \times \left(-\dfrac{9}{16}\right)$ ;
\item $\dfrac{56}{-5} \times \dfrac{30}{21} \dfrac{7}{10}$.
\end{enumerate}
\end{exercice} 




\begin{exercice}[Avec la calculatrice]
Utilise ta calculatrice pour effectuer les produits suivants et donne les résultats sous la forme d'une fraction irréductible :
\begin{enumerate}
\item $\dfrac{686}{-153} \times \dfrac{-99}{-196}$ ; 
\item $\dfrac{2,1}{12,5} \times \left(-\dfrac{6,25}{0,49}\right)$.
\end{enumerate}
\end{exercice}



\begin{exercice}
Calcule mentalement :
\begin{enumerate}
\item les trois quarts de 400 ;
\item le double de $\dfrac{-7}{15}$ ;
\item les cinq septièmes des six cinquièmes de l'unité ;
\item les $\dfrac{7}{10}$ de $\dfrac{9}{10}$.
\end{enumerate}
\end{exercice}




\begin{exercice}[Dépense]
Abdel dépense les $\dfrac{5}{12}$ de son argent de poche, puis les trois quarts de ce qu'il lui reste alors.
\begin{enumerate}
\item Quelle fraction de son argent de poche a-t-il dépensée la deuxième fois ?
\item Le montant de son argent de poche étant de 72\,€, combien a-t-il dépensé au total ?
\end{enumerate}
\end{exercice}



\begin{exercice}
Recopie et complète en utilisant $=$ ou $\neq$, en justifiant dans chaque cas :
\begin{enumerate}
\item $\dfrac{-9,1}{5,2} ... \dfrac{79,8}{-45,6}$ ;
\item $\dfrac{-5}{-3} ... \dfrac{-3,5}{2,1}$ ;
\item $\dfrac{17,36}{-22,32} ... -\dfrac{28,7}{36,9}$ ;
\item $\dfrac{-56}{-57} ... \dfrac{57}{58}$ ;
\end{enumerate}
\end{exercice}





\serie{Divisions}


\begin{exercice}Inverses
Recopie et complète les égalités suivantes et écris, dans chaque cas, trois phrases utilisant le mot « inverse(s) » :
\begin{enumerate}
\item $4 \times \dfrac{1}{...} = 1$ ;
\item $... \times 0,25 = 1$ ;
\item $\dfrac{1}{...} \times (-3) = 1$ ;
\item $... \times \left(-\dfrac{1}{15}\right) = 1$ ;
\item $\dfrac{3}{4} \times \dfrac{...}{...} = 1$ ;
\item $\dfrac{...}{-25} \times \dfrac{...}{7} = 1$ ;
\item $... \times \left(-\dfrac{8}{5}\right) = 1$ ;
\item $-0,01 \times ... = 1$
\end{enumerate}
\end{exercice}



\begin{exercice}[Ne pas confondre !]
\begin{enumerate}
\item Recopie et complète les égalités suivantes :
\[ \left(\dfrac{9}{-14}\right) \times ... = 1 et \left(\dfrac{9}{-14}\right) + ... = 0\].
Écris deux phrases, l'une utilisant le mot « opposé(s) » et l'autre, le mot « inverse(s) ».
\item Trouve deux nombres qui sont leur propre inverse. Trouve un nombre qui est son propre opposé.
\item Tous les nombres ont-ils un inverse ? Un opposé ?
\item Quel est l'opposé de l'inverse de 4 ? Quel est l'inverse de l'opposé de 4 ?
\end{enumerate}
\end{exercice}



\begin{exercice}[Inverse]
\begin{enumerate}
\item Recopie et complète le tableau ci-dessous avec des écritures fractionnaires.

\renewcommand*\tabularxcolumn[1]{>{\centering\arraybackslash}m{#1}}
\renewcommand{\arraystretch}{1.6}
\begin{Ctableau}{\linewidth}{7}{c}
\hline
$x$ & 7 & $\dfrac{-3}{5}$ & $-\dfrac{8}{9}$ & $-0,6$ & $1,25$ \\ \hline
$\dfrac{1}{x}$ & & & & & \\ \hline
\end{Ctableau}
\item Détermine l'inverse de l'inverse de chaque nombre. Que remarques-tu ?
\end{enumerate}
\end{exercice}


\begin{exercice}[Mentalement]
\begin{enumerate}
\item Effectue mentalement les calculs suivants : $16 \div 2$ ; $100 \times 0,25$ ; $16 \times 0,5$ ; $100 \div 4$.
\item Justifie les résultats égaux avec la règle de division.
\end{enumerate}
\end{exercice}


\begin{exercice}La règle
Applique dans chaque cas la règle de division puis effectue les calculs :
\begin{enumerate}
\item $\dfrac{2}{3} \div 5$ ; 
\item $\dfrac{-5}{7} \div (-4)$ ;
\item $\dfrac{5}{6} \div \dfrac{7}{-11}$ ;
\item $8 \div \dfrac{1}{8}$ ;
\item $\dfrac{-3}{2} \div \dfrac{-5}{7}$ ;
\item $\dfrac{1}{10} \div \left(-\dfrac{7}{9}\right)$.
\end{enumerate}
\end{exercice}

\begin{exercice}Trait de fraction
Écris les quotients suivants en utilisant le symbole $\div$ puis effectue le calcul :
\[ A = \dfrac{2}{\dfrac{3}{5}}  ; B = \dfrac{\dfrac{2}{3}}{5} ; C = \dfrac{\dfrac{2}{3}}{\dfrac{7}{11}}.\]
\end{exercice}


\begin{exercice}[Division et simplification]
Applique la règle de division, simplifie puis effectue les calculs et donne les résultats sous la forme d'une fraction irréductible :
\begin{enumerate}
\item $\dfrac{8}{-15} \div \dfrac{-4}{5}$ ;
\item $\dfrac{9}{10} \div (-3)$ ;
\item $\dfrac{-4}{45} \div \dfrac{16}{15}$ ;
\item $\dfrac{-5}{6} \div \left(-\dfrac{15}{18}\right)$ ;
\item $12 \div \dfrac{3}{-4}$ ;
\item $1 \div \left(\dfrac{-7}{4}\right)$.
\end{enumerate}
\end{exercice}
