\begin{exercice}[Partage]
\begin{enumerate}
\item Calcule la moitié de $\dfrac{-5}{12}$.
\item Il reste les $\dfrac{7}{8}$ d'un gâteau.

Trois amis décident de se partager équitablement ce reste : quelle fraction du gâteau aura chacun d'entre eux ?
\end{enumerate}
\end{exercice}


\begin{exercice}[Avec des lettres]
\begin{enumerate}
\item Sachant que $a = \dfrac{-2}{21}$  et $b =\dfrac{5}{-7}$, calcule :
\[	\dfrac{a}{b} ; \dfrac{b}{a} ; a \times b ; a + b \text{ et } a - b\]
Tu donneras les résultats sous la forme d'une fraction irréductible.
\item Même consigne avec $a =\dfrac{5}{24}$  et $b =-\dfrac{35}{18}$.
\end{enumerate}
\end{exercice}


\begin{exercice}
Jenny avait 145 fr. Elle a dépensé les $\dfrac{2}{5}$ de ce qu'elle avait. Combien d’argent lui reste-t-il ?
\end{exercice}


\begin{exercice}
Un salarié gagne 3900 CHF par mois. Il dépense $\dfrac{3}{20}$ de cette somme pour son loyer, $\dfrac{1}{13}$ pour les impôts et 2000 CHF pour vivre. Combien économise-t-il chaque mois ?
\end{exercice}

\columnbreak
\begin{exercice}
J'ai dépensé les $\dfrac{4}{5}$ de mon argent pour acheter un livre qui coûtait 32 CHF. Quelle somme avais-je dans mon porte-monnaie?
\end{exercice}


\begin{exercice}
Olivier a les $\dfrac{7}{16}$ des $\dfrac{2}{3}$ de l'âge de sa mère qui a 48 ans. Quel est l’âge d’Olivier ?
\end{exercice}


\begin{exercice}
Pierre dit à sa sœur pour l'impressionner: « Ce livre a coûté très cher. Je l'ai payé $\dfrac{5}{12}$ des $\dfrac{6}{5}$ de 20 CHF. » Quel est le prix du livre ?
\end{exercice}


\begin{exercice}
Alicia et Alizée ont groupé leurs économies pour s’acheter un lecteur MP3 et des DVD. 

Elles ont dépensé les $\dfrac{6}{10}$ de leur pactole pour l'achat du lecteur MP3 et les $\dfrac{6}{9}$ de ce qu'il restait pour l’acquisition des DVD. Après ces achats, il ne leur reste plus que 28 €.
\begin{enumerate}
\item De quelle somme disposaient-elles avant de faire leurs achats ? 
\item Quel est le prix du lecteur MP3 et celui des DVD ?
\end{enumerate}
\end{exercice}


\begin{exercice}
Monsieur Reesh avait 500 000 CHF dans son coffre mais Arsène Lupin est passé par là et lui a dérobé $\dfrac{3}{4}$ des $\dfrac{5}{6}$ des $\dfrac{4}{5}$ de la somme. Combien lui reste-t-il dans son coffre ?
\end{exercice}


\begin{exercice}
Lors de ses dernières vacances, Alex a dépensé les $\dfrac{3}{4}$ des $\dfrac{5}{9}$ des $\dfrac{7}{10}$ de son argent de poche qui se montait à 3000 CHF. Quelle somme lui reste-t-il ?
\end{exercice}

