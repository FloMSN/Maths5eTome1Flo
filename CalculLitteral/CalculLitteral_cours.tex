\section{Définitions}

\begin{definition}
\begin{itemize}
    \item Une expression littérale est une expression mathématique qui contient une ou plusieurs lettres.
    \item Pour tout nombre $a$, on peut écrire :
        \subitem $a \times a = a^2$	(qui se lit « $a$ au carré ») ;
		\subitem $a \times a \times a = a^3$ (qui se lit « $a$ au cube »).
	\item Un monôme est une expression de la forme $ax^n$ où $a$ est un nombre appelé coefficient du monôme et $n$ un nombre entier naturel appelé degré du monôme.
\end{itemize}
\end{definition} 

\begin{exemple*1}

$4a+1$ ; $2n^2$ ; $2ab-6$ ; $xy^2$ sont des expressions littérales

$x^2$ ; $5x^3$ ; $-12x$ ; $25x^2$ sont des monômes

$12ab$ ; $3xy$ ne sont pas des monômes
\end{exemple*1}


\begin{definition}
Un polynôme est une somme de monômes. Un polynôme avec deux termes est appelé binôme, avec trois termes, trinôme.
\end{definition} 

\begin{exemple*1}

$5x^3-3$ est un polynôme également appelé binôme

$3x^2+5x-7$ est un polynôme également appelé trinôme

$-5x^3+x^2-12x+23$ est un polynôme
\end{exemple*1}


\section{Simplification d'une expression littérale}
       
\begin{aconnaitre}
\begin{itemize}
    \item Il y a deux signes pour la multiplication : on peut utiliser indifféremment $\times$ et $\cdot$.
    \item \textbf{Pour simplifier l'écriture d'une expression littérale}, on peut supprimer le signe de multiplication devant une lettre ou une parenthèse.
\end{itemize}
\end{aconnaitre} 

\begin{remarque}
On ne peut pas supprimer le signe $\times$ entre deux nombres.
\end{remarque}

\begin{exemple*1}
Simplifie l'expression suivante : $A = -5 \times x + 7 \times (-4) \times (3 \times x -2)$.

\correction

\begin{tabular}{lcl}
$A = -5 \times x + 7 \times (-4) \times (3 \times x -2)$ & $\longrightarrow$ & On repère tous les signes $\times$. \\
$A = -5x + 7 \times (-4)(3x -2)$ & $\longrightarrow$ & On supprime les signes $\times$ placés devant une lettre \\
& &  ou une parenthèse. \\
$A = -5x -28(3x -2)$ & $\longrightarrow$ & On calcule si possible. \\
\end{tabular}

\end{exemple*1} 


\begin{methode*1}[Remplacer des lettres par des nombres]
Pour \textbf{calculer une expression littérale pour une certaine valeur des lettres}, il suffit de remplacer les lettres par ces valeurs. 
\exercice

Calcule l'expression $A = 5x(x + 2)$ pour $x = 3$.

\correction

\begin{tabular}{lcl}
$A = 5 \times x \times (x + 2)$ & $\longrightarrow$ & On replace les signes $\times$ dans l'expression $A$. \\
$A = 5 \times 3 \times (3 + 2)$ & $\longrightarrow$ & On remplace la lettre $x$ par sa valeur 3. \\
$A = 15 \times 5$ & $\longrightarrow$ & On effectue les calculs. \\
$A = 75$ & & \\
\end{tabular}

\end{methode*1}




