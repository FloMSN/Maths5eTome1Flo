\begin{enigme}[Des signes...]
\begin{enumerate}
    \item $a$, $b$ et $c$ sont trois nombres relatifs dont le produit est négatif et $b$ est le double de $a$.
    
    Quel est le signe de $c$ ?
    \item $x$, $y$ et $z$ sont trois nombres relatifs tels que :
		$x \times y$ et $x \times z$ ont le même signe ;
		$y$ et $y \times z$ ont le même signe ;
		$y$ et $x \times y \times z$ ont des signes différents.
		
	Quels sont les signes de $x$, $y$ et $z$ ?
	\item Pour quelles valeurs de $m$, le produit de $m$ par $m -1$ est-il négatif ?
	\item Donne le signe de $x -1$ en fonction des valeurs de $x$.
Étudie le signe du produit $x(x -1)(x -2)$ en fonction des valeurs de $x$.
\end{enumerate}
\end{enigme}

\vspace{2em}

\begin{enigme}[La danse des signes]
Certains nombres relatifs ont perdu leur signe et il peut manquer des signes d'opérations ! À toi de les retrouver !

$(-2) \times (... 3) -(-8) \times (... 2) = 22$ ;

$(-2) \times (... 3) -(-8) \div (... 2) = 2$ ;

$(-2) ... (... 3) -(-8) \times (... 2 ... 4) = -10$.
\end{enigme}


\vspace{2em}

\begin{enigme}[Que de signes !]
Détermine le signe du produit suivant :

$(-343) \times (-344) \times (-345) \times ... \times (-999)$.
\end{enigme}



\vspace{2em}

\begin{enigme}[Le compte est bon]
Avec les nombres proposés, retrouve les résultats annoncés !

Tu ne peux utiliser chaque nombre qu'une seule fois. Toutes les opérations sont autorisées.

Avec $-3$ ; $-5$ ; 25 ; $-100$ et 7, trouve $-650$ !

Avec $-7$ ; $-25$ ; 10 ; $-8$ et $-75$, trouve 730 !
\end{enigme}


\vspace{2em}

\begin{enigme}[Qui suis-je ?]
Ce nombre est très bizarre : que je le multiplie par $-2$ ou par $-7$, j'obtiens le même résultat ! Quel est ce nombre ?

Quand je me multiplie par moi-même, cela donne mon opposé ! Qui suis-je ?
\end{enigme}



\vspace{2em}

\begin{enigme}[Des signes, toujours des signes...]
$a$ et $b$ sont des nombres relatifs. Étudie leurs signes dans chacun des cas suivants :
\begin{itemize}
    \item $a + b$ est un nombre négatif et $a \times b$ est un nombre positif ;
    \item $a + b$ et $a \times b$ sont des nombres négatifs.
\end{itemize}
\end{enigme}


\vspace{2em}

\begin{enigme}[Les nombres négatifs dans l'histoire]
Les nombres négatifs font aujourd'hui partie de notre environnement. Nous les considérons comme des nombres à part entière.

Pourtant, leur introduction dans les mathématiques fut lente, difficile et maintes fois remise en cause. Ils naissent à travers les calculs de gains et de dettes. On attribue aux Chinois les premières utilisations de quantités négatives au premier siècle de notre ère.

\vspace{1em}


Voici ce que disait, en 1803, le mathématicien et ingénieur Lazare Carnot (1753 - 1823) à leur propos :

{\em « Pour obtenir réellement une quantité négative isolée, il faudrait retrancher une quantité effective de zéro, ôter quelque chose de rien : opération impossible. Comment donc concevoir une quantité négative isolée ? ».}

\vspace{1em}

Voici deux autres citations de mathématiciens :

\vspace{1em}

Pascal (1623 - 1662), dans ses « Pensées » :

{\em « Trop de vérité nous étonne ; j’en sais qui ne peuvent comprendre que, qui de zéro ôte 4, reste zéro. ».}

\begin{center}
    \includegraphics[width=2cm]{pascal}
    
\hfill {\footnotesize (Image : Pascal, source Wikipédia)}
\end{center}


\vspace{1em}

Arnauld (un ami de Pascal), à propos de l’égalité $\dfrac{-1}{1}=\dfrac{1}{-1}$ :

{\em « Comment un nombre plus petit pourrait-il être à un plus grand comme un plus grand à un plus petit ? ».}

\begin{enumerate}
    \item Explique ces phrases et commente-les.
    \item Et que penser de la réflexion suivante de Wallis (1616 - 1703) ?
    
{\em « a étant un nombre positif, le quotient  est infini. Comme  est plus grand, le dénominateur étant plus petit, il est plus grand que l’infini tout en étant inférieur à zéro car le résultat est négatif. »}
\end{enumerate}

\end{enigme} 

%%%%%%%%%%%%%%%%%%%%%%%%%%%%%%%%%%%%%%%%%%%%%%%%%
%%%%%%%%%%%%%%%%%%%%%%%%%%%%%%%%%%%%%%%%%%%%%%%%%