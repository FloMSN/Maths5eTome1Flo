\section{Rappels sur les opérations avec des nombres relatifs}

\subsection{Addition / soustraction de nombres relatifs}

\begin{rappel}
La \MotDefinition{valeur absolue}{} d'un nombre relatif est \textbf{le nombre sans son signe}. Sur une droite graduée, cela correspond à la distance entre l'origine et le point qui a pour abscisse ce nombre.
\end{rappel}


\begin{aconnaitre}
Pour \textbf{additionner deux nombres relatifs de même signe}, on additionne leur valeur absolue et on garde le signe commun.

Pour \textbf{additionner deux nombres relatifs de signes contraires}, on soustrait la plus petite valeur absolue de la plus grande et on prend le signe de celui qui a la plus grande valeur absolue.

\textbf{Soustraire un nombre relatif} revient à ajouter \textbf{son opposé} (\textbf{L'opposé d'un nombre relatif} est le nombre de signe contraire qui a la même valeur absolue).
\end{aconnaitre}


\begin{exemple*1}
Effectue l'addition suivante : $A = (-2) +(-3)$. 

\correction

\begin{tabular}{lcl}
$A = (-2) +(-3)$ & $\rightarrow$ & On veut additionner deux nombres relatifs de même signe. \\
$A = -(2 +3)$ & $\rightarrow$ & On additionne leur valeur absolue et on garde le signe commun : $-$. \\
$A = -5 $ & $\rightarrow$ &  On calcule. \\
\end{tabular}

\end{exemple*1}





\begin{exemple*1}
Effectue l'addition suivante : $B = (-5) +(+7)$.  

\correction

\begin{tabular}{lcl}
$B = (-5) +(+7)$ & $\rightarrow$ & On veut additionner deux nombres relatifs de signes contraires. \\
$B = +(7 -5)$ & $\rightarrow$ & On soustrait leur valeur absolue et on écrit le signe du nombre qui a \\
& & la plus grande valeur absolue $(+7)$. \\
$B = +2$ & $\rightarrow$ & On calcule. \\
\end{tabular}

\end{exemple*1}





\begin{exemple*1}
Effectue la soustraction suivante : $C = (-2) -(-3)$. 

\correction

\begin{tabular}{lcl}
$C = (-2) -(-3)$ & $\rightarrow$ &  On veut soustraire le nombre $-3$\\
$C = (-2) +(+3)$ & $\rightarrow$ &  On ajoute l'opposé de $-3$ qui est $+3$.\\
$C = +(3 -2)$ & $\rightarrow$ &  On ajoute deux nombres de signes contraires donc on soustrait \\
& & leur valeur absolue et on prend le signe du nombre qui a la plus \\
& & grande valeur absolue $(+3)$. \\
$C = +1$ & $\rightarrow$ &  On calcule.\\
\end{tabular}

\end{exemple*1}
 



\subsection{Multiplication / division de deux nombres relatifs}

\begin{aconnaitre}
Pour multiplier (ou diviser) deux nombres relatifs, on multiplie (ou on divise) leur valeur absolue et on applique la \MotDefinition{règle des signes}{} suivante :
\begin{itemize}
    \item le produit de deux nombres relatifs de \textbf{même signe} est \textbf{positif} ($"+" \times "+" = "+"$ et $"-" \times "-" = "+"$) ;
    \item le produit de deux nombres relatifs de \textbf{signes contraires} est \textbf{négatif} ($"+" \times "-" = "-"$ et $"-" \times "+" = "-"$).
\end{itemize}
\end{aconnaitre}	





\begin{exemple*1}
Effectue la multiplication : $F = (-4) \times (-2,5)$.

\correction
Le résultat est positif car c'est le produit de deux nombres relatifs de même signe (négatifs).

$F = 4 \times 2,5 \qquad F = 10$
\end{exemple*1}

\begin{exemple*1}
Effectue la division : $G = 6,8 \div (-2)$.

\correction
Le résultat est négatif car c'est le quotient de deux nombres de signes contraires (un nombre positif par un nombre négatif).

$G = -(6,8 \div 2) \qquad		G = -3,4$
\end{exemple*1}

\begin{remarque}
Multiplier un nombre relatif par $-1$ revient à prendre son opposé.
\end{remarque}


\subsection{Multiplication de plusieurs nombres relatifs}

\begin{aconnaitre}
Le produit de plusieurs nombres relatifs est : 
\begin{itemize}
    \item \textbf{positif} s'il comporte un nombre \textbf{pair} de \textbf{facteurs négatifs};
    \item \textbf{négatif} s'il comporte un nombre \textbf{impair} de \textbf{facteurs négatifs}.
\end{itemize}
\end{aconnaitre}

\begin{exemple*1}
Quel est le signe du produit : $H = -6 \times 7 \times (-8) \times (-9)$ ?

\correction
Le produit comporte trois facteurs négatifs. Or 3 est impair donc $H$ est négatif.
\end{exemple*1}


\begin{exemple*1}
Calcule le produit : $J = 2 \times (-4) \times (-5) \times (-2,5) \times (-0,8)$.

\correction
Le produit comporte quatre facteurs négatifs. Or 4 est pair donc $J$ est positif.
$J = 2 \times 4 \times 5 \times 2,5 \times 0,8$

$J = (2 \times 5) \times (4 \times 2,5) \times 0,8$

$J = 10 \times 10 \times 0,8 = 80 $
\end{exemple*1}


\section{Priorité des opérations}

\begin{aconnaitre}
Dans une suite d'opérations avec des nombres relatifs, on effectue \textbf{dans l'ordre} : d'abord les calculs entre parenthèses puis les calculs de puissances, les multiplications et divisions et enfin les additions et soustractions.
\end{aconnaitre}


\begin{exemple*1}
Effectue le calcul suivant : $M = -4 -5 \times (-2 -6)$.

\correction

\begin{tabular}{lcl}
$M = -4 -5 \times \underline{(-2 -6)}$ & $\rightarrow$ & On repère le calcul prioritaire. \\
$M = -4 \underline{-5 \times (-8)}$ & $\rightarrow$ & On effectue d'abord le calcul entre parenthèses. \\
$M = \underline{-4 + 40}$ & $\rightarrow$ & On effectue ensuite la multiplication. \\
$M = 36$ & $\rightarrow$ & On termine par l'addition. \\
\end{tabular}
\end{exemple*1}





\section{Approximation de nombres relatifs}

\subsection{Troncature}

\begin{definition}
Prendre la troncature d’un nombre à une précision donnée, c’est couper ce nombre et enlever tous les chiffres qui dépassent la précision demandée.
\end{definition}

\begin{exemple*1}
Donner la troncature à $10^{-2}$ de $349,7275$ et la troncature au dixième de $57,93$.
\correction
La troncature à $10^{-2}$ de $349,7275$ est $349,72$ et la troncature au dixième de $57,93$ est $57,9$.
\end{exemple*1}

\subsection{Valeur approchée par défaut et par excès}
 
\begin{remarque}
Lorsqu’un nombre comporte plusieurs chiffres après la virgule, on peut en donner une valeur approchée à :
$10^{-1}$ près, c’est-à-dire 1 chiffre après la virgule. C’est une valeur approchée au dixième.
$10^{-2}$ près, c’est-à-dire 2 chiffres après la virgule. C’est une valeur approchée au centième.
$10^{-3}$ près, c’est-à-dire 3 chiffres après la virgule. C’est une valeur approchée au millième.
\end{remarque}

\begin{definition}
La \MotDefinition{valeur approchée à l’unité par défaut}{} d’un nombre est le nombre entier immédiatement plus petit que notre nombre.

La \MotDefinition{valeur approchée à l’unité par excès}{} d’un nombre est le nombre entier immédiatement plus grand que notre nombre.
\end{definition}

\begin{exemple*1}
Donner les valeurs approchées par défaut et par excès, à l'unité, du nombre $12,58$.

\correction
Le nombre entier immédiatement plus petit que $12,58$ est $12$. Donc la valeur approchée par défaut à l'unité de $12,58$ est $12$. Le nombre entier immédiatement plus grand que $12,58$ est $13$. Donc la valeur approchée par excès à l'unité de $12,58$ est $13$.
\end{exemple*1}



\begin{definition}
La \MotDefinition{valeur approchée au dixième par défaut}{} d’un nombre est le nombre décimal ayant un seul chiffre après la virgule immédiatement plus petit que notre nombre.

La \textbf{valeur approchée au dixième par excès}{} d’un nombre est le nombre décimal ayant un seul chiffre après la virgule immédiatement plus grand que notre nombre.
\end{definition}

\begin{exemple*1}
Donner les valeurs approchées par défaut et par excès au dixième du nombre $8,193$.

\correction
Le nombre ayant un chiffre après la virgule et qui est immédiatement plus petit que $8,193$ est $8,1$. Donc la valeur approchée par défaut au dixième (ou à $10^{-1}$) de $8,193$ est $8,1$. Le nombre ayant un chiffre après la virgule et qui est immédiatement plus grand que $8,193$ est $8,2$. Donc la valeur approchée par excès au dixième (ou à $10^{-1}$) de $8,193$ est $8,2$.
\end{exemple*1}


\begin{remarque}
On définit de la même façon la valeur approchée par défaut ou par excès à une précision donnée : valeur approchée par défaut à 10-2 près (au centième), ou valeur approchée par excès à 10-4 près (au dix-millièmes) ...
\end{remarque}


\subsection{Arrondi}

\begin{definition}
Donner l’arrondi d’un nombre positif, c’est déterminer \textbf{une valeur approchée de ce nombre en fonction du chiffre qui suit directement sa troncature} :
\begin{itemize}
    \item Si ce chiffre est 0, 1, 2, 3 ou 4, l’arrondi correspond à la troncature.
    \item Si ce chiffre est 5, 6, 7, 8 ou 9, on rajoute 1 au dernier chiffre de sa troncature.
\end{itemize}
\end{definition}

\begin{exemple*1}
Donner les arrondis à $10^{-2}$ et $10^{-3}$ de $83,372851$. Donner la troncature au dixième et l’arrondi au dixième de $175,378$.
\correction
L'arrondi à $10^{-2}$ de $83,374851$ est $83,37$ (car le chiffre des millièmes est 4) et l'arrondi à $10^{-3}$ est $83,375$ (car le chiffre des dix-millièmes est 8).

La troncature au dixième de $175,378$ est $175,3$ et l’arrondi au dixième est $175,4$ (car le chiffre des centièmes est 7).
\end{exemple*1}
