\begin{TP}[Le bon produit]

\partie{La construction du jeu}

Avec du papier épais ou du carton, fabriquez  66 cartes à jouer.

Au stylo bleu, fabriquez les 38 cartes « facteur » :
\begin{itemize}
\item deux portent le nombre 0 ;
\item trois exemplaires pour chacun des nombres : $-9$ ; $-6$ ; $-4$ ; $-3$ ; $-2$ ; $-1$ ; 1 ; 2 ; 3 ; 4 ; 6 et 9.
\end{itemize}

Remarque : Soulignez les 6 et les 9 pour éviter de les confondre.

Au stylo rouge, fabriquez les 28 cartes « produit » : 
\begin{itemize}
\item deux portent le nombre 0 ;
\item les autres sont toutes différentes et portent les nombres : $-54$ ; $-36$ ; $-27$ ; $-24$ ; $-18$ ; $-16$ ; $-12$ ; $-9$ ; $-8$ ; 6 ; $-4$ ; $-3$ ; $-2$ ; 2 ; 3 ; 4 ; 6 ; 8 ; 9 ; 12 ; 16 ; 18 ; 24 ; 27 ; 36 et 54.
\end{itemize}

\partie{Les règles du jeu}

Chaque joueur reçoit six cartes « facteur » puis pioche une carte « produit ». Celui qui a le plus grand nombre joue en premier (en cas d'égalité, les joueurs ex-aequo piochent une deuxième carte « produit »). On tourne ensuite dans le sens des aiguilles d'une montre.

Les cartes « produit » piochées sont posées face visible. On complète de façon à en avoir 10 en tout sur la table.

Le joueur dont c'est le tour pioche une carte « produit » et la pose sur la table avec les autres. 

Si, avec deux de ses cartes facteurs, il peut obtenir un des produits visibles, il écarte les trois cartes (les deux cartes « facteur » et la carte « produit »).
S'il ne peut pas, il pioche deux cartes « facteur » et regarde à nouveau s'il peut obtenir un produit.

S'il propose une combinaison et qu'il a fait une erreur de calcul, il pioche également deux cartes « facteur ».

C'est alors au tour du joueur suivant.

Lorsqu'un joueur a écarté toutes ses cartes « facteur », il a gagné.

\end{TP}

%%%%%%%%%%%%%%%%%%%%%%%%%%%%%%%%%%%
%%%%%%%%%%%%%%%%%%%%%%%%%%%%%%%%%%%


\begin{TP}[Expressions littérales]

\partie{Résolution d'énigmes}

Dans chaque cas, retrouvez les valeurs de chacune des inconnues pour que l'égalité soit vérifiée sachant qu'elles sont données dans le désordre.

Exemple : Pour le problème :

\renewcommand*\tabularxcolumn[1]{>{\centering\arraybackslash}m{#1}}
\renewcommand{\arraystretch}{1.6}
\begin{Ctableau}{\linewidth}{6}{c}
\hline
$a(b -c) -de = -5$ & 4 & $-1$ & $-3$ & $-2$ & 1 \\ \hline
\end{Ctableau}

Une solution est : $a = -3$ ; $b = 1$ ; $c = -2$ ; $d = 4$ et $e = -1$. En effet :

$-3 \times (1 -(-2)) -4 \times (-1) = -9 + 4 = -5$.

\vspace{1em}

\textbf{Niveau 1 : trois inconnues}

\vspace{1em}

\renewcommand*\tabularxcolumn[1]{>{\centering\arraybackslash}m{#1}}
%\renewcommand{\arraystretch}{1.6}
\begin{Ctableau}{\linewidth}{4}{c}
\hline
$a + b -c = 3$ & $-2$ & 4 & $-3$ \\ \hline
$a + bc = 1$ & $-1$ & 3 & 4 \\ \hline
$a -(b -c) = 1$ & $-5$ & 2 & 8 \\ \hline
$\dfrac{a}{-b+c} = -1,5$ & 3 & 1 & $-3$ \\ \hline
$a+\dfrac{b}{c} = 3$ & $-1$ & 2 & 5 \\ \hline
\end{Ctableau}


\vspace{1em}

\textbf{Niveau 2 : quatre inconnues}

\vspace{1em}

\renewcommand*\tabularxcolumn[1]{>{\centering\arraybackslash}m{#1}}
%\renewcommand{\arraystretch}{1.6}
\begin{Ctableau}{\linewidth}{5}{c}
\hline
$(a + b)(c + d) = -60$ & $-9$ & $-4$ & $-3$ & 9 \\ \hline
$\dfrac{a-b}{c+d} = 3$ & $-9$ & $-3$ & 9 & 9 \\ \hline
$a(b -c) -d = 17$ & 4 & $-5$ & 7 & $-8$ \\ \hline
$ab -cd = 1$ & $-3$ & $-5$ & 5 & 8 \\ \hline
$a -\dfrac{b}{c} -d = -7$ & $-6$ & $-10$ & 5 & 3 \\ \hline
\end{Ctableau}

\vspace{1em}

\textbf{Niveau 3 : cinq inconnues}

\vspace{1em}

\renewcommand*\tabularxcolumn[1]{>{\centering\arraybackslash}m{#1}}
%\renewcommand{\arraystretch}{1.6}
\begin{Ctableau}{\linewidth}{6}{c}
\hline
$a(b + c) -de = 19$ &
$-1$ &
3 &
2 &
5 &
$-6$ \\ \hline
$\dfrac{a}{b+c}- \dfrac{d}{e} = -1$ & $-1$ & 3 & 8 & 2 & 4 \\ \hline
$a(b + c) -a(d -e) = 1$ & 2 & 4 & $-1$ & $-2$ & 3 \\ \hline
\end{Ctableau}



\partie{\'A vous de faire les énigmes}

Maintenant, construisez des énigmes sur le modèle précédent : deux de niveau 1, deux de niveau 2 et une de niveau 3.

Les énigmes seront ensuite rassemblées au tableau et chaque groupe devra essayer d'en résoudre le plus grand nombre possible.


\end{TP}

%%%%%%%%%%%%%%%%%%%%%%%%%%%%%%%%%%%
%%%%%%%%%%%%%%%%%%%%%%%%%%%%%%%%%%%