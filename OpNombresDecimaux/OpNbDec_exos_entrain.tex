
\serie{Additionner et soustraire}


\begin{exercice}Effectue les additions suivantes :
\begin{enumerate}
\item $(+4) +(+9)$
\item $(-2) +(+3)$
\item $(-4) +(-11)$
\item $(+1) +(-7)$
\item $(-10) +(+10)$
\item $(-40) +(+20)$
\end{enumerate}
\end{exercice}


\begin{exercice}Effectue les calculs suivants :
\begin{enumerate}
\item $9 -12$
\item $-10 -6$
\item $-2 - (-17)$
\item $-13 - (-5)$
\item $8 -1$
\item $0 -(-72)$
\end{enumerate}
\end{exercice}


\begin{exercice}Effectue les calculs suivants :
\begin{enumerate}
\item $15,7 +22,8$
\item $-51,5 +31,7$
\item $7,2 -3,1$
\item $31,2 -13,4$
\item $-2,8	- (-3,9)$
\item $-50	-12,4$
\end{enumerate}
\end{exercice}


\begin{exercice}Effectue les calculs suivants :
\begin{enumerate}
\item $5 -17$
\item $8 -21$
\item $-5 -2$
\item $-7 +11$
\item $31 -37$
\item $-2,8 -2,1$
\item $-8,3 +3,5$
\item $1,7 -3,52$
\end{enumerate}
\end{exercice}


\begin{exercice}Effectue les calculs suivants :
\begin{enumerate}
\item $13,2 +12,8$
\item $-25,5 +11,7$
\item $2,3 +(-1,5)$
\item $17,4 - (12,6)$
\item $-3,9 - (-11,1)$
\item $-100 - (+13)$
\end{enumerate}
\end{exercice}



\begin{exercice}Effectue les calculs suivants (tu peux  regrouper les termes de même signe) :

$A = 24 +8 -12 +1 -5$

$B = -14 +5 -7 -10 +13$

$C = 11 -(-5) +(-8) +1 -(+17)$

$D = -11 -(+4) +23 +(-12) -(-18)$
\end{exercice}



\begin{exercice}Effectue les calculs suivants (tu peux  regrouper les termes de même signe) :

$A = 15 +3 -6 +2 -7 $

$B = -8 +4 -5 -6 +11$

$C = 10 -(-4) +(-1) +5 -9$

$D = (-15) -(+14) +30 +(-15) -(-20)$
\end{exercice}


\begin{exercice}Regroupe les termes astucieusement puis calcule :

$A = 22 +25 +8 -25$

$B = -1,5 +5,7 -3,6 +0,3 +1,5$
\end{exercice} 



\serie{Multiplier}



\begin{exercice}Calcule mentalement :
\begin{enumerate} 
\item $(-8) \times (+2)$
\item $(-2) \times (+5)$
\item $(-4) \times (-8)$
\item $(+9) \times (+10)$
\item $(+191) \times (+0,1)$ 
\item $(-1,5) \times (+20)$
\item $(-0,25) \times (-4)$
\item $(+0,8) \times (-3)$
\item $(-3,2) \times (+4)$
\item $(-1) \times (-17)$
\end{enumerate}
\end{exercice}



\begin{exercice}Relie chaque calcul à son résultat :
\begin{colitemize}{2}
\item $(+5) \times (-7)$
\item $(-6) \times (-4)$
\item $(-4) \times (+3)$
\item $(+7)	\times (+7)$
\item $(-3) \times (-4)$
\item $(-9) \times (-3)$
\item $(-5) \times (+3)$
\item $(-8) \times (+6)$
\item $-15$
\item $-35$
\item $-12$
\item $+12$
\item $-48$
\item $+27$
\item $+24$
\item $+49$
\end{colitemize}
\end{exercice}



\begin{exercice}Calcule, sachant que $11,2 \times 2,5 = 28$ :
\begin{enumerate}
\item $11,2 \times (-2,5)$
\item $-11,2 \times (-2,5)$
\end{enumerate}
\end{exercice}



\begin{exercice}[Un produit peut en cacher un autre...]
\begin{enumerate}
\item Calcule le produit $7,5 \times 0,2$.
\item Effectue alors les calculs suivants :
\subitem $A = 7,5 \times (-0,2)$
\subitem $B = (-0,2) \times (-7,5)$
\subitem $C = (-75) \times (+0,2)$
\subitem $D = (-7,5) \times (-20)$
\end{enumerate}
\end{exercice}



\begin{exercice}Complète les « pyramides » suivantes sachant que le nombre contenu dans une case est le produit des nombres contenus dans les deux cases situées en dessous de lui :
\end{exercice}


\begin{exercice}Effectue les calculs suivants :

$A = (-4) \times (-7) \times (+5)$

$B = (-2) \times (-5) \times (-3)$

$C = (+5) \times (-1) \times (+9)$
\end{exercice}



\begin{exercice}Effectue les calculs suivants :

$A = (-2,2) \times (-10) \times (+3) \times (-0,5)$

$B = (-50) \times (-0,25) \times (+4) \times (+2)$

$C = (-4) \times (-0,1) \times (+5) \times (+4)$

$D = (-1,5) \times (+4) \times (-1) \times (+0,8) \times (-3)$

$E = (+4) \times (-10) \times (+2) \times (-1) \times (-1)$
\end{exercice}



\begin{exercice}Calcule astucieusement :

$A = (-2) \times (-1,25) \times (-2,5) \times (-8)$

$B = (-75) \times (-0,25) \times (+2) \times (+4)$

$C = (+0,01) \times (-25) \times (-13) \times 4 \times (-3)$
\end{exercice}



\begin{exercice}Calcule dans chaque cas le produit $xy$ :
\begin{enumerate}
\item $x = 5 et y = -3$
\item $x = +4 et y = -11$
\item $x = -2 et y = -5$
\item $x = -0,5 et y = -5,2$
\end{enumerate}
\end{exercice}



\begin{exercice}Complète le tableau suivant :

\renewcommand*\tabularxcolumn[1]{>{\centering\arraybackslash}m{#1}}
\renewcommand{\arraystretch}{1.6}
\begin{ltableau}{\linewidth}{6}
\hline
$a$ & $b$ & $c$ & $ab$ &  $-(ac)$ & $abc$ \\ \hline 
$-5$ & 6 & $-4$ & & &  \\ \hline
$-1$ & $-2$ & $-3$ &  & & \\ \hline
$-2,1$ & $-4$ & $+3$  & & & \\ \hline
\end{ltableau}
\end{exercice}



\begin{exercice}[Décompositions...]
\begin{enumerate}
\item Trouve toutes les façons de décomposer le nombre $-20$ en produit de deux nombres entiers relatifs.
\item Trouve toutes les façons de décomposer le nombre 24 en produit de trois nombres entiers relatifs.
\end{enumerate}
\end{exercice} 



\serie{Diviser}



\begin{exercice}Calcule mentalement :
\begin{enumerate}
\item $64 \div (-4)$
\item $48 \div (-8)$
\item $-27 \div (-3)$
\item $72 \div (+9)$
\item $-71 \div ( -1)$
\item $-42 \div 7$
\item $(-42) \div (-6)$
\item $125 \div (-5)$
\item $(-7) \div (+7)$
\item $(-29) \div (+1)$
\end{enumerate}
\end{exercice}



\begin{exercice}Calcule mentalement :
\begin{enumerate}
\item $(-100) \div (+25)$
\item $(-42) \div (-4)$
\item $(+54) \div (-3)$
\item $(+55) \div (+5)$
\item $(-24) \div (-5)$
\item $(-13) \div (-10)$
\end{enumerate}
\end{exercice}


\begin{exercice}Pour chaque fraction, trouve l'écriture la plus simple possible : 

Exemple : $\dfrac{-2}{+9}=-\dfrac{2}{9}$
\begin{enumerate}
\item $-\dfrac{+4}{+5}$
\item $-\dfrac{-1}{-5}$
\item $\dfrac{7}{-3}$
\item $-\dfrac{-8}{11}$
\item $-\dfrac{1}{-10}$
\item $-\dfrac{5}{-15}$
\end{enumerate}
\end{exercice}



\begin{exercice}Sans calculatrice, donne l'écriture décimale de chacun des nombres suivants :
\begin{enumerate}
\item $-\dfrac{3}{-10}$
\item $-\dfrac{-64}{-8}$
\item $\dfrac{-50}{+100}$
\item $\dfrac{-3}{-2}$
\end{enumerate}
\end{exercice}



\begin{exercice}Utilise ta calculatrice pour donner l'écriture décimale des nombres suivants :
\begin{enumerate}
\item $\dfrac{-5}{-40}$
\item $-\dfrac{172}{-5}$
\item $-\dfrac{-125}{-625}$
\item $\dfrac{-0,235}{+0,8}$
\end{enumerate}
\end{exercice}



\begin{exercice}Dans chaque cas, calcule le quotient de $x$ par $y $:
\begin{enumerate}
\item $x = -15$ et $y = -3$
\item $x = +64$ et $y = -8$
\item $x = -36$ et $y = 12$
\item $x = -2,4$ et $y = 1,2$
\item $x = y = -2,3$
\item $x = 0$ et $y = -5$
\end{enumerate}
\end{exercice}



\begin{exercice}Complète le tableau suivant et donne le résultat sous forme décimale :

\renewcommand*\tabularxcolumn[1]{>{\centering\arraybackslash}m{#1}}
\renewcommand{\arraystretch}{1.6}
\begin{ltableau}{\linewidth}{6}
\hline
$a$ & $b$ & $c$ & $a \div b$ &  $-(b) \div c$ & $c \div (-a)$ \\ \hline 
$-5$ & 4 & $-4$ & & &  \\ \hline
$-2,5$ & $-1$ & $+20$ &  & & \\ \hline
$+8$ & $-4$ & $-0,5$  & & & \\ \hline
$-2,4$ & $-1,2$ & $-24$  & & & \\ \hline
\end{ltableau}
\end{exercice} 



\serie{Calculs variés}



\begin{exercice}Pour chacun des calculs suivants, indique s'il s'agit d'une somme ou d'un produit puis donne le résultat :
\begin{enumerate}
\item $-4 \times (+9)$
\item $-3 -(+8)$
\item $-7 +(-5)$
\item $+3 \times (-7)$
\item $-8 +(+6)$
\item $+9 \times (+3)$
\item $-5 -(-16)$
\item $-11 \times (-4)$
\end{enumerate}
\end{exercice}



\begin{exercice}Sans les calculer, donne le signe de chacun des calculs suivants :
\begin{enumerate}
\item $(-4) \times (-12)$
\item $(+15) +(-22)$
\item $(-45) -(-51)$
\item $(-37) \times (+51)$
\item $(+7) \times (+8)$
\item $(-7) +(+8)$
\item $(-3,12) \times (-2,5)$
\item $(-3,17) -(+3,7)$
\end{enumerate}
\end{exercice}



\begin{exercice}Calcule mentalement :
\begin{enumerate}
\item $8 \times (-8)$
\item $-22 +(-6)$
\item $-14 \times 3$
\item $-5 -(+17)$
\item $(-34) +(-19)$
\item $-15 \times (-5)$
\end{enumerate}
\end{exercice}



\begin{exercice}Calcule mentalement :
\begin{enumerate}
\item $(-4) \times (-2,5)$
\item $(+3,5) +(-2,2)$
\item $(-3,9) +(-5,4)$
\item $(-3) \times (+4,2)$
\item $(+3,4) \times (-2)$
\item $(-7,15) -(-2,2)$
\item $(-3,12) \times (-10)$
\item $(-0,7) -(+1,17)$
\end{enumerate}
\end{exercice}



\begin{exercice}Complète les « pyramides » suivantes sachant que le nombre contenu dans une case est le produit des nombres contenus dans les deux cases situées en dessous de lui :
\begin{center}
    \begin{tikzpicture}
    \node[name=a,businessman,minimum size=1.5cm] at (-2,0) {};
    \node[ellipse callout, draw,yshift= 1cm,xshift=.5cm, callout absolute pointer={(a.mouth)},    font=\footnotesize] {Deux pyramides ici !};

    \end{tikzpicture}
\end{center}
\end{exercice}



\begin{exercice}Effectue les calculs suivants en soulignant, à chaque étape, le calcul en cours :

$A = 7 +(-6) \times (-6)$

$B = 13 -(+3) \times (-4) -8$

$C = -30 \div (-9 +15)$

$D = -3 -9 \times (-3)$

$E = -3 \times 6 \times (-2 +8)$
\end{exercice}



\begin{exercice}Effectue les calculs suivants en soulignant, à chaque étape, le calcul en cours :

$A = -22 +(13 -5) \times (-5)$

$B = (-2) \times (-8) +2 \times (-20) \div 4$

$C = -28 +(5 -2) \times (-4)$

$D = 7 \times (-7) +3 \times (-25) \div (-5)$

$E = -3,1 \times (-6) +(-2,3 -7,7)$

$F = 150 \div (-3 -9 \times 3)$
\end{exercice}



\begin{exercice}Effectue les calculs suivants :

$A = 18 -[(-7 +13 ) -( 5 -23) -( 15 -8 )] -13$

$B = 28 \div (-3 \times 4 -2)$

$C = [13 -(-2)] \times 2 +5$

$D = (- 4 - 4 \times (- 4)) \div (2 \times (- 3))$

$E = [(-3) \times (-1 -4) +(-12) \div 4] \times (-3) +5$
\end{exercice}




\begin{exercice}Calcule les expressions suivantes :

$A = \dfrac{11}{2-5}$

$B = \dfrac{-6-3}{2+7}$

$C = \dfrac{-2-(-4)}{6-7}$
\end{exercice}



\begin{exercice}Effectue les calculs suivants :

$A = 16 \div (-2 \times 5 +2)$

$B = - 42 \div (-6 \times 3 +25)$

$C = [-8 -(-3)] \times 3 +4$

$D = (4 - 3 \times (- 4)) \div 4 -2$

$E = 62 \div 4 -(42 -2 \times 6) -7 +9 \times 2$
\end{exercice}




\begin{exercice}[Vocabulaire]
\begin{enumerate}
\item Traduis les phrases suivantes par un calcul :
    \begin{itemize}
    \item La somme du produit de 4 par $-5$ et de $-6$.
    \item Le produit de la somme de 7 et de $-8$ par la somme de 8 et de $-2$.
    \end{itemize}
\item Effectue ces calculs.
\end{enumerate}
\end{exercice}


\begin{exercice}[Vocabulaire (bis)]
\begin{enumerate}
\item Traduis les phrases suivantes par un calcul :
    \begin{itemize}
    \item La différence du quotient de 16 par $-2$ et de $-5$.
    \item Le quotient de la différence de 13 et de $-2$ par la somme de $-2$ et de $-3$.
    \end{itemize}
\item Effectue ces calculs.
\end{enumerate}
\end{exercice}



\begin{exercice}[Vocabulaire (ter)]
\begin{enumerate}
\item Traduis les phrases suivantes par un calcul :
    \begin{itemize}
    \item Le produit du quotient de 24 par $-4$ par la somme de 3 et de $-6$.
    \item La différence du quotient de $-63$ par 7 et de  la somme de $-4$ et de $-5$.
    \end{itemize}
\item Effectue ces calculs.
\end{enumerate}
\end{exercice}



\begin{exercice}[Vocabulaire (dans l'autre sens !)]

\begin{enumerate}
\item Traduis les expressions mathématiques suivantes par des phrases :

Exemple : $(-2) \times 3 +1$ se traduit par :
\og La somme du produit de $(-2)$ par 3 et de 1.\fg
    \begin{itemize}
    \item $A = 5 \times (-7) +3$
    \item $B = 3 + \dfrac{2}{-4}$
    \item $C = 7 - 4 \times (-10)$
    \item $D = (2 - 3) \times (-1 -2)$
    \item $E = \dfrac{1-7}{2+5}$
    \item $F = -2 +(-6) \times (-6) -9$
    \end{itemize}
\item Effectue ces calculs.
\end{enumerate}
\end{exercice}


\begin{exercice}Complète le tableau suivant :

\renewcommand*\tabularxcolumn[1]{>{\centering\arraybackslash}m{#1}}
\renewcommand{\arraystretch}{1.6}
\begin{ltableau}{\linewidth}{5}
\hline
$a$ & $b$ & $c$ & $ab-c$ &  $(a-b)c$ \\ \hline 
$2$ & 3 & $5$ & &  \\ \hline
$-1$ & $5$ & $6$  & & \\ \hline
$3$ & $-5$ & $-7$  & & \\ \hline
$-8$ & $2$ & $-6$  & & \\ \hline
\end{ltableau}
\end{exercice}



\begin{exercice}Pour $a = 3$, $b = -4$, $c = -5$ et $d = 7$, calcule les expressions suivantes :

$A = a -b +c$

$B = 2a -3b$

$C = ac -bd$

$D = -5ac +bd$

$E = 2(a  - b) +d$

$F = 5(b  -a) \div d$
\end{exercice}



\begin{exercice}Complète le tableau suivant :

\renewcommand*\tabularxcolumn[1]{>{\centering\arraybackslash}m{#1}}
\renewcommand{\arraystretch}{1.6}
\begin{ltableau}{\linewidth}{6}
\hline
$a$ & $b$ & $c$ & $ab$  & $(-ac)$ & $abc$ \\ \hline 
$-5$ &  & $+4$ & 10 & &  \\ \hline
 &  & $2$ & &  $-12$ & $-36$ \\ \hline
\end{ltableau}
\end{exercice}




\begin{exercice}Supprime les parenthèses dans chaque expression puis calcule sans calculatrice :

$A = [(-5) +6 - (-1) - 7] - [(-5) +6 - (-1) - 7]$

$B = [(-5) +6 - (-1) - 7] - [(-5) +6 - (-1) +7]$

$C = -18,1 +2,8 -7 +(-2,8 +18,1 -7)$

$D = 18,1 +2,8 -7 -(2,8 +18,1 +7)$
\end{exercice}



\begin{exercice}Effectue les calculs suivants en respectant les priorités :
\begin{enumerate}
\item $36 +(6 - 2)3 : 2 +102 - 4 +5 \times 2 - 1$
\item $(9 - 2)2 : 7 - 1 +43 : 16 - 5$
\item $92 : 3 +5 - (5 - 2) +36 +4$
\item $(12 - 2)2 : 20 +56 : 23 +25 - 5 : 5$
\item $25 +5 : 5 - 22 +6 +2 \times 3 - 1$
\item $451 - 12 \times 5 +60 \times 4$
\end{enumerate}
\end{exercice}



\begin{exercice}Effectue les calculs suivants en respectant les priorités :
\begin{enumerate}
\item $3^2 - 2^4 \times (9 - 10 +5) - 2 \times (5 +3^2)$
\item $2^4 : 4 - 2^3 \times 5^2 - 4^3 : 2^3 +36$
\item $10^2 - 4^2 \times (3^2 - 12 +2) +3$
\item $(3$$4 +2 - 3^0 +5^2) - 4 \times (- 3 +6^2 +1^4)$
\item $(3^2 - 2^3) \times 5 +2 \times (3 +1^5)$
\item $(3 \times 5 +2^2) \times 2 +(6 - 3 \times 2)^2$ 
\item $2 \times ( - 8) +8^2 $
\item $(350 : 10) \times (4^2 - 12)$
\end{enumerate}
\end{exercice}


\serie{Approximation}


\begin{exercice}On considère le nombre  $349,2856$.
\begin{enumerate}
\item Donner les troncatures : à l'unité, à $10^{-2}$ près, à $0,001$ près et à la centaine de ce nombre.
\item Donner les arrondis : à l'unité, au dixième, à $10^{-3}$ près, à la dizaine, à $0,01$ près de ce nombre.
\end{enumerate}
\end{exercice}



\begin{exercice}Recopier et compléter le tableau suivant :

\renewcommand*\tabularxcolumn[1]{>{\centering\arraybackslash}m{#1}}
\renewcommand{\arraystretch}{1.6}
\begin{cltableau}{\linewidth}{5}
\hline
nombre & valeur arrondie à l'unité & tron\-ca\-ture à $10^{-1}$ & valeur approchée à $10^{-2}$ par excès & valeur approchée à $10^{-1}$ par défaut \\ \hline
 $385,1829$ & & & & \\ \hline
 $17,4351$ & & & & \\ \hline
 $0,8796$ & & & & \\ \hline
\end{cltableau}
\end{exercice}




\begin{exercice}Complète les tableaux suivants avec les valeurs arrondies et les troncatures à l’unité, à $10^{-1}$ près, et au millième des nombres donnés.

\renewcommand*\tabularxcolumn[1]{>{\centering\arraybackslash}m{#1}}
\renewcommand{\arraystretch}{1.6}
\begin{cltableau}{\linewidth}{4}
\hline
nombre & valeur arrondie à l'unité & valeur arrondie à $10^{-1}$ & valeur arrondie au millième \\ \hline
$445,2541$ & & &  \\ \hline
$225,1247$ & & &  \\ \hline
$222,29143$ & & &  \\ \hline
$5,1452$ & & &  \\ \hline
$0,1726$ & & &  \\ \hline
$4,9273$ & & &  \\ \hline
$3,4216$ & & &  \\ \hline
$12,9214$ & & &  \\ \hline
\end{cltableau}

\vspace{1em}

\renewcommand*\tabularxcolumn[1]{>{\centering\arraybackslash}m{#1}}
\renewcommand{\arraystretch}{1.6}
\begin{cltableau}{\linewidth}{4}
\hline
nombre & tron\-ca\-ture à l'unité & tron\-ca\-ture à $10^{-1}$ & tron\-ca\-ture au millième \\ \hline
$445,2541$ & & &  \\ \hline
$225,1247$ & & &  \\ \hline
$222,29143$ & & &  \\ \hline
$5,1452$ & & &  \\ \hline
$0,1726$ & & &  \\ \hline
$4,9273$ & & &  \\ \hline
$3,4216$ & & &  \\ \hline
$12,9214$ & & &  \\ \hline
\end{cltableau}
\end{exercice}




\begin{exercice}Complète les tableaux suivants avec les valeurs approchées par excès et par défaut à l'unité, au dixième et au millième des nombres donnés.

\renewcommand*\tabularxcolumn[1]{>{\centering\arraybackslash}m{#1}}
\renewcommand{\arraystretch}{1.6}
\begin{cltableau}{\linewidth}{4}
\hline
nombre & valeur approchée par excès à l'unité & valeur approchée par défaut à l'unité & valeur approchée par excès à $10^{-1}$ \\ \hline
$23,1785$ & & &  \\ \hline
$193,3591$ & & &  \\ \hline
$18,5555$ & & &  \\ \hline
$0,0101$ & & &  \\ \hline
$385,854$ & & &  \\ \hline
$10,985$ & & &  \\ \hline
\end{cltableau}

\vspace{1em}

\renewcommand*\tabularxcolumn[1]{>{\centering\arraybackslash}m{#1}}
\renewcommand{\arraystretch}{1.6}
\begin{cltableau}{\linewidth}{4}
\hline
nombre & valeur approchée par défaut à $10^{-1}$ & valeur approchée par excès au millième & valeur approchée par défaut au millième \\ \hline
$23,1785$ & & &  \\ \hline
$193,3591$ & & &  \\ \hline
$18,5555$ & & &  \\ \hline
$0,0101$ & & &  \\ \hline
$385,854$ & & &  \\ \hline
$10,985$ & & &  \\ \hline
\end{cltableau}
\end{exercice}



\begin{exercice}Donne la troncature à $10^{-1}$, l'arrondi à 
$10^{-2}$ et la valeur approchée par excès à l'unité des nombres suivants :
\begin{enumerate}
\item $12,345$
\item $67,891$
\item $0,0001$
\item $958,12$
\item $25,0102$
\item $5,5$
\end{enumerate}
\end{exercice} 



\begin{exercice}Donne, à l'aide de ta calculatrice, l'arrondi à l'unité de chacun des nombres suivants, comme dans l'exemple :

Exemple : $A =\dfrac{-153}{23}$. La calculatrice donne $A \approx -6,652173913$. On a donc : $-7 < A < -6$. L'arrondi à l'unité de $A$ est $-7$ car $A$ est plus proche de $-7$ que de $-6$.

\vspace{.5em}

$B =\dfrac{39}{-9}$ 

$C = \dfrac{-17}{-7}$ 

$D = \dfrac{-28}{51}$ 
\end{exercice}
