\begin{TP}[Des phrases et des expressions littérales]

\vspace{1em}\textbf{1\up{ère} Partie : Programmes de calculs}\vspace{1em}


\item Voici un programme de calculs :

\begin{center}
    \begin{minipage}{.6\linewidth}
        \begin{oldalgorithme}
Choisir un nombre ;
Lui ajouter 1 ;
Multiplier le résultat par 3 ;  
Ajouter 4 au résultat.%
        \end{oldalgorithme}
    \end{minipage}
\end{center}

\vspace{.5em}

Appliquez ce programme au nombre 2.

\item Omar a appliqué ce programme pour un certain nombre et a trouvé 28. Laura dit alors que, pour retrouver le nombre de départ, il suffit de \og remonter \fg le programme de calculs en partant de la fin. Expliquez ce que veut dire Laura. Combien trouve-t-elle ?
\item \label{DisFC01} Voici un autre programme de calculs :


\begin{center}
    \begin{minipage}{.6\linewidth}
        \begin{oldalgorithme}
Choisir un nombre ;
Lui ajouter 3 ;
Multiplier le résultat par 2 ;
Ajouter le nombre du départ ;
Enlever 6 au résultat.%
        \end{oldalgorithme}
    \end{minipage}
\end{center}

\vspace{.5em}


Appliquez ce programme au nombre 5.

\item La méthode de Laura fonctionne-t-elle encore ? Pourquoi ?
\item Marc a trouvé une méthode pour trouver directement le nombre de départ, connaissant le résultat de la fin du programme. Expliquez cette méthode.
\item Construisez quatre programmes différents (dont au moins deux comme le programme de la question \ref{DisFC01}) qui transforment un nombre en un nombre quatre fois plus grand. 


 
\vspace{1em}\textbf{2\up{e} Partie : En français dans le texte}\vspace{1em}


Avant l'invention de l'algèbre, les mathématiciens utilisaient le langage courant pour écrire certaines propriétés, on pouvait énoncer la règle suivante :
\og Soit un nombre entier. Si on ajoute le nombre qui le précède et le nombre qui le suit, on obtient le double du nombre \fg.
\item En appelant $n$ le nombre, écrivez une égalité qui traduit cette phrase.
\item Procédez de la même façon pour le texte suivant : \og La différence des carrés de deux nombres entiers consécutifs est égale au double du plus petit augmenté de 1. \fg.
\item Développez et réduisez : $(x - 1)(x + 1)$ Chaque membre du groupe écrit en français un texte décrivant l'égalité obtenue.
\item Comparez vos textes. Écrivez ensemble celui qui vous paraît le plus clair.

\end{TP}




\newpage

\begin{TP}[Le Mistigri des expressions littérales]

   
\vspace{1em}\textbf{1\up{ère} Partie : Préparation du jeu}\vspace{1em}  
       
\item On commence par préparer un jeu de vingt et une cartes. Sur chaque carte est écrit une expression ci-dessous :

\vspace{1em}

\renewcommand*\tabularxcolumn[1]{>{\centering\arraybackslash}m{#1}}
\begin{ttableau}{\linewidth}{3}
\hline
$3x(2x - 5)$ & $5(x - 3)$ & $(x - 1)(x + 2)$ \\ \hline
$(x - 3)(x - 2)$ & $5(x - 5)$ & $7x + 14$ \\ \hline
$3x - (2x - 1)$ & $x - (x + 1)$ & $4(3x^2 - 2x + 1)$ \\ \hline
$x^2 + x - 2$ & $x^2 - 5x + 6$ & $7(x + 2)$ \\ \hline
$x + 1$ & $6x^2 - 15x$ & $x^2 - 1$ \\ \hline
$(x + 1)(x - 1)$ & $12x^2 - 8x + 4$ & $5x - 15$ \\ \hline
$5x - 25$ & $x^2 + 1$ & $-1$ \\ \hline
\end{ttableau}

\vspace{1em}


\item Préparez ensemble une feuille contenant côte à côte les expressions qui sont égales. Une expression n'est égale à aucune autre : c'est le Mistigri. La feuille servira de référence en cas de désaccord pendant la partie mais elle devra rester cachée. Les joueurs n'ont pas le droit de l'utiliser.


\vspace{1em}\textbf{2\up{e} Partie : On joue}\vspace{1em}



\item Un joueur distribue toutes les cartes en commençant par son voisin de gauche. 
\item Chaque joueur regarde dans son jeu s'il possède une paire, autrement dit deux cartes comportant des expressions égales. Tout au long de la partie, si un joueur a une paire, il l'écarte de son jeu en la posant face visible sur la table. Les autres joueurs vérifient que la paire est correcte.
\item Le donneur prend une carte au hasard au joueur qui est à sa gauche, puis regarde s'il  possède une nouvelle paire qu'il écarte alors de son jeu. Puis le joueur à la droite du donneur prend une carte au hasard au donneur et ainsi de suite.
\item Le gagnant est le joueur qui se débarrasse le premier de toutes ses cartes. Le perdant est celui qui a le mistigri en main lorsque toutes les paires ont été formées.

\vspace{.5em}
Remarque : il est fortement conseillé aux joueurs de s'aider d'un brouillon.


\vspace{1em}\textbf{3\up{e} Partie : Fabrication d'un nouveau jeu}\vspace{1em}


\item À vous maintenant de créer un jeu de Mistigri sur le même principe que précédemment mais avec d'autres expressions.
\item Jouez avec votre jeu mais cette fois-ci sans utiliser de feuille contenant les \og paires \fg. À la fin de votre partie, échangez votre jeu avec un autre groupe avant de rejouer.

\end{TP}
