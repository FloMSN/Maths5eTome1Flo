\serie{Puissance d'un nombre}

\begin{exercice}[]
Voici une liste de mots : exposant, puissance, facteurs, produit. Recopie chaque phrase en la complétant par le mot qui convient.

\begin{colenumerate}{1} 
\item $3^7$ se lit « 3 ... 7 ».
\item $5^4$ est le ... de quatre ... tous égaux à 5.
\item 8 est l'... de $6^8$.
\item Le ... de six ... égaux s'écrit sous la forme d'une ... d'... 6.
\end{colenumerate} 
\end{exercice}

\begin{exercice}[D'une écriture à l'autre]

\begin{colenumerate}{1} 
\item Écris en toutes lettres : $3^4$ ; $2^3$ ; $7,1^9$ et $(-4)^2$.
\item Écris en expressions mathématiques :
    \subitem huit puissance neuf 
    \subitem quatre au cube 
    \subitem trois puissance cinq 
    \subitem sept au carré
\end{colenumerate} 
\end{exercice}

\begin{exercice}[]\label{Pex1}
Recopie et complète chaque expression par l'exposant manquant :

\begin{colenumerate}{1} 
\item $4 \times 4 \times 4 \times 4 \times 4 \times 4 \times 4 \times 4 \times 4 = 4^{...}$
\item $(-5) \times (-5) \times (-5) \times (-5) \times (-5) = (-5)^{...}$
\item $0,1 \times 0,1 \times 0,1 = 0,1^{...}$
\end{colenumerate} 
\end{exercice}

\begin{exercice}[]
Décompose chaque nombre comme dans l'exercice \RefExercice{Pex1} :

\begin{colenumerate}{3} 
\item $9^4$
\item $2^3$
\item $5^7$
\item $(-7)^5$
\item $5,3^4$
\item $(-0,8)^3$ 
\end{colenumerate} 
\end{exercice}



\begin{exercice}[]
Quels sont les nombres négatifs :

\begin{colenumerate}{3} 
\item $(-6)^4$
\item $6^8$
\item $-132^{51}$
\item $(-12)^{15}$
\item $(-3)^7$
\item $(-3,6)^{100}$
\item $-(-35)^7$
\item $-87^4$
\item $-(-13^8)$
\end{colenumerate} 
 
\end{exercice}


\begin{exercice}[Puissance de 1 ou de $-1$]

Calcule :

\begin{colenumerate}{4} 
\item $1^{12}$
\item $1^0$
\item $(-1)^8$
\item $(-1)^0$
\item $-1^7$
\item $-1^6$
\item $(-1)^9$
\item $-1^0$
\end{colenumerate} 
 
\end{exercice}

\begin{exercice}[Exposant 0 ou 1]

Calcule :

\begin{colenumerate}{4} 
\item $4^0$
\item $0,5^1$
\item $(-6)^0$
\item $1,2^1$
\item $0,5^1$
\item $-5^1$
\item $(-1,8)^1$
\item $-7^0$
\end{colenumerate} 
 
\end{exercice}

\begin{exercice}[]
Décompose puis donne l'écriture décimale en calculant à la main :

\begin{colenumerate}{4} 
\item $2^4$
\item $-2^4$
\item $(-2)^4$
\item $7^2$  
\item $(-3)^4$
\item $-3^4$
\item $(-6)^3$
\end{colenumerate} 
 
\end{exercice}

\begin{exercice}[]
Donne l'écriture décimale en calculant à la calculatrice :

\begin{colenumerate}{4} 
\item $2^{14}$ 
\item $17^{7}$ 
\item $8^{11}$
\item $12^3$
\item $-3^{10}$
\item $(-11)^8$
\item $(-4)^5$
\item $-6^4$
\end{colenumerate} 
 
\end{exercice}

\begin{exercice}[]
Écris les nombres suivants sous la forme d'un produit :

\begin{colenumerate}{1} 
\item de puissances de 2 et de 5 :
    \subitem $A = 2 \times 2 \times 5 \times 5 \times 5 \times 2 \times 2 \times 5 \times 5$
    \subitem $B = 25 \times 10 \times 5 \times 8$
    \subitem $C = 625 \times 512$
\item de puissances de 2, de 3 et de 7 :
    \subitem $D = 2 \times 2 \times 2 \times 3 \times 7 \times 7$ 
    \subitem $E = 32 \times 21 \times 12$ 
    \subitem $F = 12 \times 21 \times 49$
    \subitem $G = 42$
\end{colenumerate} 
\end{exercice}

\begin{exercice}[]
Écris sous la forme d'un produit :

\begin{colenumerate}{1} 
\item de puissances de 2 et de 5 :
    \subitem $A =\dfrac{2\times 2\times 2\times 5 \times 5}{2\times 2 \times 5}$ 
    \subitem $B =\dfrac{25}{4} \times \dfrac{64}{8}$ 
\item de puissances de 2, de 3 et de 7 :
    \subitem $C =\dfrac{2\times 3\times 3\times 7 \times 7}{2\times 3 \times 7 \times 7}$ 
    \subitem $D =\dfrac{2^6\times 3^4\times 7^2}{49\times 32 \times 27}$ 
\end{colenumerate} 
\end{exercice}

\begin{exercice}[]
Écris sous la forme $a^n$, où $a$ est un nombre relatif et $n$ est un entier naturel.

\begin{colenumerate}{2} 
\item $5^2 \times 5^4$
\item $6^5\times 6^8$
\item $3^4 \times 5^4$
\item $-4\times (-4)^7$ 
\item $7^{-5}\times 7$
\item $-2^3\times (-2)^5$
\item $\left(\dfrac{2}{3}\right)^3 \times \left(\dfrac{2}{3}\right)^5$
\end{colenumerate} 
\end{exercice}

\begin{exercice}[]
Écris sous la forme $a^n$, où $a$ est un nombre relatif et $n$ est un entier relatif.
\begin{colenumerate}{3} 
\item $\dfrac{3^8}{3^4}$
\item $\dfrac{6^5}{3^5}$
\item $\dfrac{20^6}{4^6}$
\end{colenumerate} 
 
\end{exercice}

\begin{exercice}[]
Écris sous la forme $a^n$, où $a$ est un nombre relatif et $n$ est un entier naturel.

\begin{colenumerate}{3} 
\item $\left(2^4\right)^3$ 
\item $\left(\left(-5\right)^3\right)^2$
\item $\left(-4^7\right)^8$
\end{colenumerate} 
\end{exercice}

\begin{exercice}[]
Écris sous la forme d'une seule puissance.
\begin{colenumerate}{2} 
\item $A=8^2 \times 8^3 \times 8^7$
\item $B=11^8 \times \dfrac{11^7}{11^4}$
\item $C=\dfrac{\left(-3\right)^6 \times \left(-3\right)^8}{\left(-3\right)^7}$
\end{colenumerate} 
\end{exercice}

\begin{exercice}[]
Recopie et complète.

\begin{colenumerate}{2} 
\item $3^4 \times 3.... = 3^9$
\item $\dfrac{2^6}{2^{...}}=2^5$
\item $4^{...} \times 4^3 = 4^3$
\item $\left(5^{...}\right)^6 = 5^{18}$
\item $\left(\left(-3\right)^2\right)^{...}=\left(-3\right)^{13}$
\end{colenumerate} 
\end{exercice}


\serie{Calculs avec des puissances}




\begin{exercice}[]
Calcule, sans calculatrice, les expressions suivantes :
$A = 3 \times 2^4 + 5 \times 4^3$
$B = 1 + 10 + 10^2 + 10^3 + 10^4 + 10^5$
$C = 1 -3^2 \times (-5)^2$
$D = 2^3 \times (-9) + 3^3 -(5^2 + 2^1)$
\end{exercice}

\begin{exercice}[]

Calcule les expressions suivantes en utilisant ta calculatrice :

\begin{colenumerate}{2} 
\item $25^3-\left(5 + 11\right)^5$
\item $\dfrac{\left(2+7\right)^5}{5-(-2)}$
\item $\left(\dfrac{-3}{8}\right)^4$
\end{colenumerate}
\end{exercice}

\begin{exercice}[]
Écris sous la forme d'une puissance :

\begin{colenumerate}{3} 
\item $3^4 \times 3^2$
\item $4^3 \times 4^{-5}$
\item $(-5)^4 \times (-5)^3$
\item $\dfrac{2^4}{2^5}$
\item $\left(7^2\right)^3$
\item $7^5 \times 2^5$
\item $8^3 \times 4^3$
\end{colenumerate} 
\end{exercice}



\begin{exercice}[]
Calcule astucieusement :
\begin{colenumerate}{2} 
\item $A = 2^4 \times 0,026 \times 5^4$
\item $B = 5^2 \times 2^2 \times 84$
\item $C = 2^3 \times 5^3 \times 2 500$
\item $D = 2^6 \times 36 \times 5^5$
\end{colenumerate}
\end{exercice}



\serie{Puissances de 10}


\begin{exercice}[]
Donne l'écriture décimale des nombres :

\begin{colenumerate}{4} 
\item $10^4$
\item $10^6$ 
\item $10^8$ 
\item $10^0$
\item $10^5$
\item $-10^0$
\item $(-10)^1$
\item $(-10)^{10}$
\end{colenumerate} 
\end{exercice}

\begin{exercice}[]
Écris à l'aide d'une puissance de 10 :

\begin{colenumerate}{1} 
\item 10 000 ; 10 000 000 ; 1 000 000 ; 1 000.
\item cent ; cent mille ; un milliard ; mille milliards.
\end{colenumerate} 
\end{exercice}

\begin{exercice}[Produit de puissances]
Exprime sous la forme d'une puissance de 10 :

\begin{colenumerate}{2} 
\item $10^5 \times 10^7$
\item $10^4 \times 10^{12}$
\item $10^8 \times 10^9$
\item $10^1 \times 10^3 \times 10^2$
\item $10 \times 10^5$
\item $10^6 \times 10^0$
\end{colenumerate} 
\end{exercice}

\begin{exercice}[Quotient de puissances]
Exprime sous la forme d'une puissance de 10 :

\begin{colenumerate}{2} 
\item $\dfrac{10^5}{10^4}$
\item $\dfrac{10^7}{10^2}$
\item $\dfrac{10^3}{10}$
\item $\dfrac{10^4}{10^0}$
\end{colenumerate} 
\end{exercice}

\begin{exercice}[Puissance de puissances]
Exprime sous la forme d'une puissance de 10 :

\begin{colenumerate}{3} 
\item $\left(10^3\right)^7$
\item $\left(10^8\right)^2$
\item $\left(10^0\right)^7$
\end{colenumerate} 

\end{exercice}

\begin{exercice}[Méli-mélo]
Écris chaque expression sous la forme d'une puissance de 10 :

\begin{colenumerate}{2} 
\item $\left(10^9\right)^4$
\item $\dfrac{10^9}{10^4}$
\item $10^{12} \times 10^0 \times 10^5$
\item $\dfrac{10^6}{10^6}$
\item $\dfrac{10^{12}\times 10^5}{10^9}$
\item $10^9 \times 10^{12}$
\item $\dfrac{10^0}{10^8}$
\item $\left(10^3\right)^1$
\item $\left(10^{10}\right)^0$
\item $\dfrac{10^{21}}{10^4 \times 10^{17}}$
\end{colenumerate}
\end{exercice}

\begin{exercice}[]
Recopie et complète par l'exposant manquant. Tu indiqueras sur ton cahier l'opération que tu as effectuée pour trouver ce nombre :

\begin{colenumerate}{2} 
\item $10^4 \times 10^{...} = 10^7$
\item $10^{...} \times 10^7 = 10^{13}$
\item $10^8 \times 10^{...} = 10^8$
\item $10^8 \times 10^{...} = 10^9$
\end{colenumerate} 

\end{exercice}

\begin{exercice}[]
Complète les phrases suivantes :

\begin{colenumerate}{1} 
\item Lorsque je multiplie un nombre positif par $10^3$, j'obtiens un résultat ... fois plus ... que le nombre de départ.
\item Lorsque je divise un nombre positif par $10^2$, j'obtiens un résultat ... fois plus ... que le nombre de départ.
\end{colenumerate}
\end{exercice}
