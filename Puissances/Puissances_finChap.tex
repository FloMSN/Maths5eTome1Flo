\begin{enigme}[Curiosité...]

Montre que la différence $10^3 -6^3$ est un carré (c'est-à-dire qu'elle peut s'écrire $n^2$, $n$ étant un entier) et que la différence $10^2 -6^2$ est un cube (c'est-à-dire qu'elle peut s'écrire $m^3$, $m$ étant un entier).

En fait, 6 et 10 sont les deux plus petits nombres qui sont tels que la différence de leurs cubes est un carré et la différence de leurs carrés, un cube !
\end{enigme}

\vspace{2em}

\begin{enigme}[Calcul impossible ?]
6 103 515 625 est une puissance de 5 et 16 777 216, une puissance de 2 : avec ta calculatrice, trouve lesquelles !

Calcule $6 103 515 625 \times 16 777 216$ sans utiliser la calculatrice cette fois !!
\end{enigme}

\vspace{2em}

\begin{enigme}[Je cherche !]
Quel est le chiffre des unités de $13^1$ ?

Celui de $13^2$ ? De $13^3$ ? De $13^4$ ? De $13^5$ ?
Quel est le chiffre des unités de $13^{2000}$ ?
\end{enigme}