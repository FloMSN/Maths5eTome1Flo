\section{Puissances entières d'un nombre relatif}

\subsection{Notations $a^n$}

\begin{definition}
Pour tout nombre entier $n$ positif non nul, pour tout nombre relatif $a$ :
\[ a^n = \underbrace{a \times a \times ... \times a}_{\text{$n$ facteurs}} \]
$a^n$ (lu « \textbf{a puissance n} ») est appelé \MotDefinition{puissance}{} $n$-ième de $a$ et $n$ est appelé l'\MotDefinition{exposant}{}.
\end{definition}


\begin{remarque}
Par convention : $a^0=1$. De plus, on a : $a^1=a$
\end{remarque}

\begin{exemple}
Donne l'écriture décimale des nombres : $2^4$ et $10^3$.
\correction
$2^4 = 2 \times 2 \times 2 \times 2 = 16$

$10^3 = 10\times10\times10 = 1 000$
\end{exemple}


\begin{exemple}
Écris sous la forme d'une puissance les expressions : $3^2 \times 3^3$ et $\dfrac{2^5}{2^3}$.
\correction
$3^2 \times 3^3  = (3 \times 3) \times (3 \times 3 \times 3) = 35$

$\dfrac{2^5}{2^3}=\dfrac{2\times2\times2\times2\times2}{2\times2\times2}=2^2$
\end{exemple} 





\subsection{Signe d'une puissance}

\begin{exemple*1}
Calculer les puissances suivantes :
\begin{itemize}
    \item $2^3 =$ 
    \item $5^2 =$ 
    \item $(-3)^4 =$ 
    \item $(-3)^3 =$ 
    \item $(-4)^2 =$ 
    \item $(-4)^3 =$ 
\end{itemize}
\end{exemple*1}

À partir des exemples ci-dessus, on peut conjecturer la propriété suivante : 
 Propriété 

\begin{propriete}
Pour tout nombre entier relatif $n$,
\begin{itemize}
    \item Si $a$ est positif alors $a^n$ est positif.
    \item Si $a$ est négatif alors $a^n$ est : \subitem positif lorsque l'exposant $n$ est pair,
			       \subitem et négatif lorsque l'exposant $n$ est impair.
\end{itemize}
\end{propriete}

\begin{exemple*1}
Quel est le signe de $A = (-3)^4$ et de $B = (-2)^5$ ?
\correction
Comme $-3$ est négatif et l'exposant 4 est pair, $A$ est un nombre positif.

Comme $-2$ est négatif et l'exposant 5 est impair, $B$ est un nombre négatif.
\end{exemple*1}
Exemple : 



\subsection{Utiliser les formules sur les puissances}


\begin{aconnaitre}
Pour tout nombre relatif $a$ non nul et pour tous nombres entiers relatifs $m$ et $p$ :
\[ a^m \times a^p = a^{m+p} \qquad ; \qquad \dfrac{a^m}{a^p}=a^{m-p} \qquad \text{et} \qquad (a^m)^p = a^{m\times p} .\]
\end{aconnaitre}

\begin{exemple*1}
Écris les expressions sous la forme $a^n$, où $a$ est un nombre relatif et $n$ un entier relatif.

$A=5^7 \times 5^4$

$B=\dfrac{(-2)^6}{(-2)^5}$

\correction

$A = 5^7 \times 5^4 = 5^{7+4}=5^{11}$

$B=\dfrac{(-2)^6}{(-2)^5}=(-2)^{6-5}=(-2)^1=-2$
\end{exemple*1}


\begin{exemple*1}
Écris le nombre $C=\dfrac{(-2)^4\times 4^5}{8^2}$ sous la forme d'une puissance de 2.

\correction

\begin{tabular}{lcl}
$C=\dfrac{(-2)^4\times 4^5}{8^2}$ & $\longrightarrow$ & On remplace 4 par 22 et 8 par 23. \\
$C=\dfrac{(-2)^4\times 4^5}{8^2}$ & $\longrightarrow$ & On remarque que (- 2)4 = 24. \\
$C=2^{4+10-6}$ & $\longrightarrow$ & On applique les règles sur les puissances. \\
$C=2^8$ & $\longrightarrow$ & On donne l'écriture demandée par l'énoncé.\\
\end{tabular}
\end{exemple*1}



\begin{aconnaitre}
Pour tous nombres relatifs $a$ et $b$ non nuls et pour tout nombre entier relatif $n$ :
\[ (a\times b)^n = a^n \times b^n  \qquad \text{et} \qquad \left(\dfrac{a}{b}\right)^n = \dfrac{a^n}{b^n} \]
\end{aconnaitre}


\begin{exemple*1}
Écris les expressions suivantes sous la forme $a^n$, où $a$ est un nombre relatif non nul et $n$ un entier relatif.

$D=2^3 \times 5^3$

$E=\dfrac{15^5}{5^5}$

$F=(-6)^5 \times \left(\dfrac{1}{3}\right)^5$

\correction

$D=2^3 \times 5^3 = (2\times 5)^3 = 10^3$

$E=\dfrac{15^5}{5^5} = \left(\dfrac{15}{5}\right)^5 = 3^5$

$F=(-6)^5 \times \left(\dfrac{1}{3}\right)^5 = \left(-6\times \dfrac{1}{3}\right)^5 = (-2)^5 = -2^5$

\end{exemple*1}





\section{Puissances de 10}

\subsection{Notations $10^n$}

\begin{definition}
Pour tout nombre entier $n > 0$ : $10^n=\underbrace{10 \times 10 \times ...\times 10}_{n \text{ facteurs}}=1\underbrace{0...0}_{n \text{ zéros}}$ et $10^0=1$. 

\end{definition}

\begin{exemple*1}
Écris les nombres 1 000, 10 000 000 et 100 000 sous la forme d’une puissance de 10.

\correction

$1 000 = 10^3$

$10 000 000 = 10^7$

$100 000 = 10^5$
\end{exemple*1}


\subsection{Multiplication par une puissance de 10}

\begin{aconnaitre}
Soit $n$ un nombre entier positif non nul.

Multiplier un nombre par $10^n$ revient à décaler la virgule de \textbf{$n$ rangs vers la droite} (on complète par des zéros si nécessaire).
\end{aconnaitre}

\begin{exemple*1}
Donne l'écriture décimale des nombres $208,641 \times 10^2$ et $-37,1 \times 10^4$.

\correction

$208,641 \times 10^2 = 20 864,1$ 

$-37,1 \times 10^4 = -371 000$
\end{exemple*1}


\begin{exemple*1}
Par combien faut-il multiplier 7,532 pour obtenir 75 320 ?

\correction

Pour passer de 7,532 à 75 320, on décale la virgule de \textbf{4 rangs vers la droit}e donc il faut multiplier 7,532 par $10^4$  pour obtenir 75 320.
\end{exemple*1}




\subsection{Calcul avec des puissances de 10}



On considère \textbf{deux nombres entiers relatifs $m$ et $p$}. Les règles de calcul avec les puissances de 10 sont les mêmes qu'avec les nombres relatifs mais elles sont très souvent utilisées : 

\begin{aconnaitre}[Règle de calcul avec deux puissances de 10]
\[ 10^m \times 10^p = 10^{m + p} \qquad ; \qquad \dfrac{10^m}{10^p}=10^{m-p} \qquad ; \qquad \left(10^m\right)^p = 10^{m\times p} \]
\end{aconnaitre} 
		
\begin{exemple*1}

Donne l'écriture décimale du nombre $A = 10^4 \times 10^3$.

\correction

$A = 10^4 \times 10^3 = 10^{4+3} = 10^7 = 10 000 000$
\end{exemple*1}



\begin{exemple*1}
Écris le nombre $B =\dfrac{10^5}{10^2}$ sous la forme d'une seule puissance de 10.

\correction

\begin{tabular}{lcl}
$B = 10^{5-2}$ & $\longrightarrow$ & On applique la règle du quotient de deux puissances de 10. (Attention aux signes moins !) \\
$B = 10^3$ & $\longrightarrow$ & On donne l'écriture demandée par l'énoncé. \\
\end{tabular}
\end{exemple*1}



\begin{exemple*1}
Écris le nombre $C =\left(10^3\right)^7 \times \left(10^2\right)^3$ sous la forme d'une seule puissance de 10.

\correction

\begin{tabular}{lcl}
$C = 10^{3 \times 7} \times 10^{2 \times 3}$ & $\longrightarrow$ & On applique la règle des puissances de puissance de 10. \\
$C = 10^{21} \times 10^6$ & $\longrightarrow$ & On effectue les multiplications sur les exposants. \\
$C = 10^{21 + 6}$ & $\longrightarrow$ & On applique la règle du produit de deux puissances de 10. \\
$C = 10^{27}$ & $\longrightarrow$ & On donne l'écriture demandée par l'énoncé. \\
\end{tabular}

\end{exemple*1}



\begin{remarque}
Attention : il n'y a pas de règle avec l'addition ou la soustraction !
\end{remarque}



\begin{exemple*1}
Donne l'écriture décimale des nombres $F = 10^3 + 10^2$ et  $G = 10^4 - 10^1$.
\end{exemple*1}