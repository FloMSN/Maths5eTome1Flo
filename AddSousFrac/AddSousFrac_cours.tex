\section{Réduction de quotients au même dénominateur}

\begin{aconnaitre}
\begin{tabular}{p{.6\linewidth}|p{.35\linewidth}}
Pour additionner ou soustraire des fractions, il faut les mettre au \textbf{même dénominateur}. Pour cela, on utilise la règle suivante : \textbf{Si on multiplie ou si on divise} le numérateur et le dénominateur d'un quotient par \textbf{un même nombre non nul} alors on obtient \textbf{un quotient égal}. & Pour tous nombres $a$, $b$ et $k$ 
où $b$ et $k$ sont non nuls : \[ \dfrac{a\times k}{b \times k} = \dfrac{a}{b} \quad \text{et} \quad \dfrac{a \div k}{b \div k} = \frac{a}{b} \] \\
\end{tabular}
\end{aconnaitre}
 



\begin{exemple*1}

Réduis les quotients $\dfrac{2}{9}$ et $\dfrac{5}{12}$ au même dénominateur.

\correction

\begin{tabular}{lcl}
Multiple de 9 : 9, 18, 27, \textbf{36}, 45, 54,... & &\\
Multiple de 12 : 12, 24, \textbf{36}, 48, 60,... & $\longrightarrow$ & On cherche un multiple commun non nul aux \\
Un multiple commun de 9 et 12 est 36. & & dénominateurs (le plus petit possible). \\
C'est aussi le plus petit. &  &  \\
$\dfrac{2}{9} = \dfrac{2 \times \boldsymbol{4}}{9 \times \boldsymbol{4}} = \dfrac{8}{36}$ et $\dfrac{5}{12} = \dfrac{5\times \boldsymbol{3}}{12 \times \boldsymbol{3}} = \dfrac{15}{36}$ & $\longrightarrow$ & On détermine les écritures fractionnaires ayant \\
& &  36 pour dénominateur. \\
\end{tabular}
\end{exemple*1}


\begin{exemple*1}

Compare les quotients $\dfrac{2}{7}$ et $\dfrac{3}{8}$.

\correction

Les dénominateurs 7 et 8 n'ont aucun diviseur commun autre que 1.

Le plus petit multiple commun est $7 \times  8 = 56$, donc $\dfrac{2 \times \boldsymbol{8}}{7 \times \boldsymbol{8}} = \dfrac{16}{56}$ et $\dfrac{3 \times \boldsymbol{7}}{8 \times \boldsymbol{7}} = \dfrac{21}{56}$.

Rappel : deux fractions qui sont au même dénominateur sont rangées dans le même ordre que leur numérateur : Comme $16 < 21$ alors $\dfrac{16}{56}<\dfrac{21}{56}$ et finalement $\dfrac{2}{7}<\dfrac{3}{8}$.
\end{exemple*1}



\section{Addition ou soustraction}

\begin{aconnaitre}
\begin{tabular}{p{.6\linewidth}|p{.35\linewidth}}
Pour \textbf{additionner (ou soustraire)} des nombres en écriture fractionnaire \textbf{ayant le même dénominateur}, on additionne (ou on soustrait) les numérateurs et on garde le dénominateur commun. & Pour tous nombres $a$, $b$ et $c$ où $b$ est non nul : \[ \dfrac{a}{b}+\dfrac{c}{b}=\dfrac{a+c}{b} \] \\
\end{tabular}
\end{aconnaitre}



\begin{remarque}
Si les nombres en écriture fractionnaire n'ont pas le même dénominateur, il faut les réduire au même dénominateur.
\end{remarque}

\begin{exemple*1}

Calcule l'expression $A =-1+\dfrac{13}{30}-\dfrac{-11}{12}$.

\correction

\begin{tabular}{lcl}
Multiples de 30 : 30 ; \textbf{60} ; 90 ; 120... & & \\
Multiples de 12 : 12 ; 24 ; 36 ; 48 ; \textbf{60}... & $\longrightarrow$ & On cherche le plus petit multiple commun non nul à 30 et 12. \\
$A =\dfrac{-1 \times 60}{1 \times 60}+\dfrac{13 \times 2}{30 \times 2}+\dfrac{11 \times 5}{12 \times 5}$ & $\longrightarrow$ & On détermine le signe de chaque quotient et on réduit les \\
& & quotients au même dénominateur 60. \\
$A =\dfrac{-60}{60}+\dfrac{26}{60}+\dfrac{55}{60}=\dfrac{-60+26+55}{60}$ & $\longrightarrow$ & On additionne les numérateurs et on garde le dénominateur. \\
$A = \dfrac{21}{60}=\dfrac{7 \times 3}{20 \times 3}= \dfrac{7}{20}$ & $\longrightarrow$ & On simplifie si possible. \\
\end{tabular}

\end{exemple*1}



