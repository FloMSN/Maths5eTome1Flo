\serie{Comparaison}

\begin{exercice}[Signes]

Donne le signe des nombres suivants :
$\dfrac{-5,2}{4,23}$ ; $\dfrac{5}{-2,1}$ ; $\dfrac{472}{23}$ ; $\dfrac{-8,9}{-45}$ ; $-\dfrac{12}{13}$ ; $-\dfrac{11}{-5,2}$.
\end{exercice}

\begin{exercice}

Indique les nombres égaux parmi ceux de la liste ci-dessous :
$\dfrac{-8}{9}$ ; $-\dfrac{8}{9}$ ; $\dfrac{-8}{-9}$ ; $-\dfrac{8}{-9}$ ; $\dfrac{8}{-9}$ ; $-\dfrac{-8}{9}$ ; $\dfrac{8}{9}$.
\end{exercice}

\begin{exercice}[Encadrement]

\begin{colenumerate}{1} 
\item On considère la fraction $\dfrac{56}{21}$.

Effectue la division euclidienne de 56 par 21 et déduis-en un encadrement de la fraction par deux nombres entiers consécutifs.
\item Encadre $\dfrac{-89}{15}$ puis $\dfrac{47}{59}$ par deux nombres entiers consécutifs.
\item Encadre respectivement $\dfrac{-47}{25}$ et $\dfrac{13}{-4}$ par deux nombres entiers consécutifs et déduis-en la comparaison de ces deux fractions.
\item Peux-tu appliquer la même méthode pour comparer $\dfrac{25}{3}$ et $\dfrac{90}{11}$ ?
\end{colenumerate} 
 
\end{exercice}

\begin{exercice}[Avec des valeurs approchées]
Soient deux nombres : $a = \dfrac{816}{577}$ et $b = \dfrac{577}{408}$.

\begin{colenumerate}{1} 
\item Donne la valeur arrondie de $a$ et celle de $b$ au millième. Peux-tu en déduire la comparaison de $a$ et de $b$ ?
\item Donne des valeurs approchées de $a$ et $b$ qui permettent de les comparer. Compare $a$ et $b$.
\end{colenumerate} 
 
\end{exercice}

\begin{exercice}[Égalités]

Recopie et complète chacune des égalités suivantes :

\begin{colenumerate}{2} 
\item $\dfrac{...}{-5} = \dfrac{10}{20}$
\item $\dfrac{2}{3} = \dfrac{...}{27}$
\item $\dfrac{-15}{45} = \dfrac{-5}{...}$
\item $\dfrac{...}{-18} = \dfrac{7}{6}$
\item $3 = \dfrac{...}{4}$
\item $-2,1 = -\dfrac{21}{...}$
\end{colenumerate} 
 
\end{exercice}

\begin{exercice}

Dans chaque cas, à partir des égalités données et en utilisant seulement les quatre nombres qui apparaissent, écris toutes les égalités d'écritures fractionnaires possibles :

\begin{colenumerate}{2} 
\item $7 \times (-8) = -4 \times 14$
\item $-3 \times (-1) = 2 \times 1,5$
\item $2,1 \times 12 = 9 \times 2,8$
\item $-4 \times 9 = 12 \times (-3)$
\end{colenumerate} 
 
\end{exercice}

\begin{exercice}[Égalité ?]
Recopie et complète en utilisant = ou $\neq$, en justifiant dans chaque cas :

\begin{colenumerate}{1} 
\item $\dfrac{-9,1}{5,2} \quad ... \quad \dfrac{79,8}{-45,6}$ 
\item $\dfrac{-5}{-3} \quad ... \quad \dfrac{-3,5}{2,1}$
\item $\dfrac{17,36}{-22,32} \quad ... \quad -\dfrac{28,7}{36,9}$
\item $\dfrac{-56}{-57} \quad ... \quad \dfrac{57}{58}$
\end{colenumerate} 
\end{exercice}


\begin{exercice}[Avec un dénominateur entier positif]

Réécris chacune des écritures fractionnaires suivantes avec un dénominateur entier positif :
$\dfrac{4}{-5}$ ; $\dfrac{-8}{-7}$ ; $-\dfrac{5,2}{-7}$ ; $\dfrac{7}{-2,1}$ ; $\dfrac{8,2}{0,12}$ ; $-\dfrac{-1}{-3,54}$.
\end{exercice}



\begin{exercice}[Même dénominateur positif]

\begin{colenumerate}{1} 
\item Recopie et complète la phrase suivante :

\og Deux nombres en écriture fractionnaire de même dénominateur positif sont rangés... \fg.
\item Compare les nombres suivants :

$\dfrac{-7,5}{3} \text{ et } \dfrac{-7,49}{3}$

$\dfrac{4,05}{2,1} \text{ et } \dfrac{4,2}{2,1}$

$-\dfrac{0,74}{5} \text{ et } \dfrac{-0,7309}{5}$

$\dfrac{8}{-5,23} \text{ et } \dfrac{-7,9}{5,23}$
\end{colenumerate} 
\end{exercice}

\begin{exercice}[Avec le même numérateur]

\begin{colenumerate}{1} 
\item Recopie et complète la phrase suivante :

\og Deux nombres positifs en écriture fractionnaire de même numérateur sont rangés... \fg
\item Compare les nombres suivants :

$\dfrac{3,5}{8,2} \text{ et } \dfrac{3,5}{8,15}$

$-\dfrac{-1}{6} \text{ et } \dfrac{1}{5,7}$
\end{colenumerate} 
\end{exercice}

\begin{exercice}[Avec le même numérateur (bis)]

Compare les nombres suivants en commençant par comparer leurs opposés :

\begin{colenumerate}{2} 
\item $\dfrac{1}{-5}$ et $\dfrac{1}{-7}$
\item $\dfrac{-3}{8}$ et $\dfrac{-3}{8,2}$
\item $-\dfrac{5,23}{14,5}$ et $\dfrac{-5,23}{14,6}$
\item $\dfrac{-7,5}{0,23}$ et $\dfrac{75}{-2,4}$
\end{colenumerate} 
\end{exercice}




\begin{exercice}

Dans chaque cas, réécris les nombres avec le même dénominateur positif puis compare-les :

\begin{colenumerate}{2} 
\item $\dfrac{-5}{4}$ et $\dfrac{-9}{8}$ 
\item $\dfrac{2,7}{-9}$ et $\dfrac{-1}{3}$ 
\item 3 et $-\dfrac{20,9}{-7}$ 
\item $-\dfrac{2}{11}$ et $\dfrac{-5}{33}$ 
\item $\dfrac{7}{2,5}$ et $\dfrac{20,5}{7,5}$ 
\item $\dfrac{13}{-27}$ et $\dfrac{-79}{162}$
\end{colenumerate} 
\end{exercice}

\begin{exercice}[Multiple commun]

\begin{colenumerate}{1} 
\item Quels sont les dix premiers multiples de 12 ? Ceux de 18 ? Déduis-en le plus petit multiple non nul commun à 12 et 18, puis un dénominateur commun positif des fractions :

\[ \dfrac{-7}{12} \quad \text{et} \quad \dfrac{-11}{18} \]

Compare alors ces deux nombres.

\item La méthode précédente permet-elle de trouver rapidement un dénominateur commun aux nombres :
\[ \dfrac{8}{11} \quad \text{et} \quad \dfrac{10}{13} \quad \text{?} \]

Comment en trouver un alors rapidement ? Compare ces deux nombres.
\end{colenumerate} 
\end{exercice}

\begin{exercice}

Dans chaque cas, réécris les nombres avec le même dénominateur positif, puis compare-les :

\begin{colenumerate}{2}
\item $\dfrac{-5}{8}$ et $\dfrac{-3,8}{6}$ 
\item $\dfrac{14}{5}$ et $\dfrac{20}{7}$ 
\item $\dfrac{3}{-50}$ et $-\dfrac{4}{75}$ 
\item $\dfrac{54,5}{0,27}$ et $\dfrac{-2,62}{-0,13}$
\end{colenumerate} 
\end{exercice}

\begin{exercice}

Compare en justifiant :

\begin{colenumerate}{2} 
\item $-\dfrac{12}{18}$ et $\dfrac{399}{-300}$ 
\item $\dfrac{2}{57}$ et $\dfrac{1}{28,4}$ 
\item $\dfrac{-75}{11}$ et $\dfrac{31}{-15}$ 
\item $\dfrac{-5}{6}$ et $-\dfrac{15}{14}$ 
\item $\dfrac{6}{13}$ et $\dfrac{29}{65}$ 
\item $\dfrac{-3}{22}$ et $\dfrac{4,5}{33}$
\end{colenumerate} 
\end{exercice}

\begin{exercice}[Dans l'ordre]

\begin{colenumerate}{1} 
\item Range les nombres suivants dans l'ordre croissant sans utiliser de valeurs approchées :
\[ \dfrac{7}{-15} ; \dfrac{7}{3} ; \dfrac{490}{420} ; \dfrac{-5}{12} ; \dfrac{-24}{-18} ; 2,5\]
\item Range les nombres suivants dans l'ordre décroissant
\[ \dfrac{-29}{100} ; \dfrac{7}{-25} ; -0,285 ; -\dfrac{1}{5} ; \dfrac{13}{-50} ; 0 ; \dfrac{-1}{2,5} \]
\end{colenumerate} 
\end{exercice}

\begin{exercice}[Trajet]
Quatre amis font un voyage en trois jours. Le premier jour, ils parcourent 40\,\%\ du trajet total ; le deuxième jour, un quart et le dernier jour, $\dfrac{7}{20}$ du trajet total.

Quel jour ont-ils parcouru la plus grande distance ?

Peux-tu calculer la distance parcourue chaque jour ?
\end{exercice}




\serie{Additions, soustractions}



\begin{exercice}[La règle]
Calcule les sommes et les différences suivantes en respectant les étapes :
\[ \dfrac{-4}{5} + \dfrac{7}{5} = \dfrac{... + ...}{...} = \dfrac{...}{...} \]

\begin{colenumerate}{2} 
\item $\dfrac{9}{7} + \dfrac{-8}{7}$
\item $\dfrac{5,2}{41} + \dfrac{8,56}{41}$
\item $\dfrac{-5}{3} + \dfrac{-6}{3}$
\item $-\dfrac{7}{15} - \dfrac{7}{15}$
\item $\dfrac{56}{57} - \dfrac{58}{57}$
\item $\dfrac{-1}{3} - \dfrac{2}{3}$
\item $\dfrac{-5}{14} - \dfrac{-2}{14}$
\item $\dfrac{1}{8} - \dfrac{9}{8}$
\item $\dfrac{5}{12} + \dfrac{11}{12} - \dfrac{7}{12}$
\item $-\dfrac{1}{25} - \dfrac{-11}{25} + \dfrac{-8}{25}$
\end{colenumerate} 
\end{exercice}

\begin{exercice}[Dénominateurs positifs]

Calcule en réécrivant dans chaque cas les fractions avec le même dénominateur positif :

\begin{colenumerate}{2} 
\item $\dfrac{8}{-5} + \dfrac{7}{5}$
\item $\dfrac{-4}{-15} + \dfrac{1}{-15}$
\item $\dfrac{5}{6} - \dfrac{7}{-6}$
\item $\dfrac{-9}{17} + \dfrac{1}{-17}$
\end{colenumerate} 
 

\end{exercice}

\begin{exercice}[Même dénominateur]
Écris les nombres suivants, si c'est possible, sous la forme $\dfrac{a}{30}$, où $a$ est un nombre décimal relatif :
\[ \dfrac{3}{10} ; \dfrac{1}{-3} ; -2 ; \dfrac{2,1}{0,6} ; \dfrac{-18}{90} ; \dfrac{1}{7} ; \dfrac{1}{-60} \]
\end{exercice}

\begin{exercice}[Avec un multiple]

\begin{colenumerate}{1} 
\item On remarque que $3 \times 8 = 24$ ;

Calcule : $\dfrac{-5}{24}+\dfrac{1}{3}$ en écrivant les fractions avec le même dénominateur positif.

\item Combien de quart(s) faut-il pour faire une unité ? Calcule $1 - \dfrac{5}{4}$ en écrivant les fractions avec le même dénominateur positif.
\item Complète : $-3 = \dfrac{...}{1} = \dfrac{... \times ...}{1 \times ...} = \dfrac{...}{8}$ .
Calcule $-3 + \dfrac{5}{-8}$ en écrivant les fractions avec le même dénominateur positif.
\end{colenumerate} 
\end{exercice}




\begin{exercice}

Effectue les calculs suivants en détaillant les étapes :

\begin{colenumerate}{2} 
\item $\dfrac{5}{6} + \dfrac{-1}{3}$
\item $\dfrac{7}{9} - \dfrac{1}{-27}$
\item $-\dfrac{8}{5} + \dfrac{23}{50}$
\item $\dfrac{45}{15} - \dfrac{7}{3}$
\item $\dfrac{4}{11} + 2$
\item $\dfrac{8}{-91} + \dfrac{-1}{7}$
\item $\dfrac{5}{2} - \dfrac{-45}{4} + \dfrac{2}{8}$
\item $4 - \dfrac{5}{-49} + \left(-\dfrac{8}{7}\right)$
\end{colenumerate}
\end{exercice}

\begin{exercice}[Trouver un dénominateur commun]

\begin{colenumerate}{1} 
\item Écris la liste des premiers multiples de 8 puis celle des premiers multiples de 6. Trouve le plus petit multiple non nul commun à 8 et 6.
 
Utilise alors ce nombre pour écrire les fractions $\dfrac{-5}{7}$ et $\dfrac{7}{6}$ avec le même dénominateur positif et calcule : $\dfrac{-5}{7}+\dfrac{7}{6}$.
\item Cette méthode permet-elle de trouver rapidement un dénominateur commun pour calculer : $\dfrac{5}{17}+\dfrac{1}{50}$ ?

Quel dénominateur commun choisir alors ? Calcule cette somme.
\end{colenumerate} 
\end{exercice}



\begin{exercice}

Effectue les calculs suivants en détaillant les étapes :

\begin{colenumerate}{2} 
\item $\dfrac{-7}{50} + \dfrac{2}{75}$
\item $\dfrac{1}{5} + \dfrac{-2}{3}$
\item $\dfrac{1}{12} - \dfrac{1}{9}$
\item $\dfrac{4}{18} + \dfrac{5}{27}$
\item $\dfrac{17}{-24} + \left(-\dfrac{5}{36}\right)$
\item $\dfrac{3}{16} - \dfrac{-1}{12}$
\item $\dfrac{8}{-17} - \left(-\dfrac{1}{15}\right)$
\item $\dfrac{2}{5} + \dfrac{-2}{15} - \dfrac{7}{12}$
\end{colenumerate} 
\end{exercice}

\begin{exercice}

Effectue les calculs suivants en détaillant les étapes et donne les résultats sous la forme d'une fraction irréductible :

\begin{colenumerate}{2} 
\item $\dfrac{42}{75} - \left(-\dfrac{22}{30}\right)$
\item $\dfrac{85}{4} + \dfrac{25}{-5}$
\item $\dfrac{-12}{25} -8$
\item $-\dfrac{14}{27} + \dfrac{-5}{108}$
\item $\dfrac{9}{-55} - \dfrac{-7}{44}$
\item $\dfrac{-9}{-18} - \dfrac{5}{30} + \left(-\dfrac{9}{6}\right)$
\item $\dfrac{1}{15} + \left(-\dfrac{1}{18}\right)$
\item $\dfrac{3}{-7} + \dfrac{2}{5}-\dfrac{4}{3}$
\end{colenumerate} 
 
\end{exercice}



\begin{exercice}[Héritage]
Après de longues négociations, il a été convenu que Léa héritera de deux quinzièmes de la fortune de son oncle du bout du monde ; Florian, d'un neuvième de cette fortune ; Jean et Justine se partageront équitablement le reste.

Quelles seront les parts respectives de Jean et Justine ?
\end{exercice}

\begin{exercice}[Opposés] 
Complète les égalités suivantes et écris, dans chaque cas, trois phrases utilisant le mot \og opposé(s) \fg :

\begin{colenumerate}{2} 
\item $\dfrac{-2}{5} + ... =0$
\item $... + \dfrac{7}{-8} = 0$
\item $... + \dfrac{-12}{-8} = 0$
\item $\left(-\dfrac{4}{5}\right) + \dfrac{9,6}{12} = ...$
\end{colenumerate} 
\end{exercice}



\begin{exercice}[Avec des lettres]
On donne : $a = \dfrac{-8}{28}$ ; $b = \dfrac{1}{35}$ et $c = \dfrac{45}{-21}$.

\begin{colenumerate}{1} 
\item Calculer $a - b + c$ et $b - a - c$.
\item Que remarques-tu ?
\end{colenumerate} 
\end{exercice}




\serie{Des équations avec des fractions}





\begin{exercice}

Résous les équations suivantes :

\begin{colenumerate}{2} 
\item $x - \dfrac{5}{4} = \dfrac{4}{3}$
\item $x + \dfrac{7}{3} = \dfrac{5}{7}$
\item $x - \dfrac{5}{8} = \dfrac{3}{12}$
\item $\dfrac{1}{3} - x = -\dfrac{2}{9}$
\item $\dfrac{5}{18} - x = \dfrac{11}{45}$
\item $x - \dfrac{12}{25} = -\dfrac{11}{15}$
\end{colenumerate} 
\end{exercice}

\begin{exercice}[Équations du type $ax = b$]

Résous les équations suivantes :

\begin{colenumerate}{3} 
\item $\dfrac{x}{5} = \dfrac{3}{4}$
\item $\dfrac{x}{7} = \dfrac{7}{6}$
\item $\dfrac{x}{11} = -\dfrac{2}{13}$
\item $\dfrac{x}{-8} = \dfrac{8}{9}$
\item $-\dfrac{x}{12} = \dfrac{7}{3}$
\item $\dfrac{7x}{2} = \dfrac{1}{4}$
\item $\dfrac{2x}{9} = -\dfrac{7}{27}$
\item $\dfrac{-3x}{7} = \dfrac{7}{8}$
\item $\dfrac{-11}{9}x = \dfrac{-1}{5}$
\end{colenumerate} 
\end{exercice}

\begin{exercice}[Équations du type $ax + b = c$]

Résous les équations suivantes :

\begin{colenumerate}{2} 
\item $\dfrac{7}{9} y + 5 = 8$
\item $\dfrac{1}{16} x - 2 = \dfrac{5}{8}$
\item $\dfrac{1}{4} x - \dfrac{3}{8} = \dfrac{2}{3}$
\item $\dfrac{3}{7} y - \dfrac{5}{35} = -\dfrac{8}{14}$
\end{colenumerate} 
\end{exercice}

\begin{exercice}

Résous les équations suivantes :

\begin{colenumerate}{2} 
\item $\dfrac{X}{3} = \dfrac{x}{4} - \dfrac{6}{5}$
\item $\dfrac{5x}{8} - \dfrac{3}{10} = \dfrac{7x}{40}$
\item $\dfrac{2x}{7} + \dfrac{3}{14} = \dfrac{x}{7} - \dfrac{1}{14}$
\item $\dfrac{2}{5} x - \dfrac{1}{9} = \dfrac{3}{9} x + \dfrac{4}{5}$
\end{colenumerate} 
\end{exercice}






\serie{Problèmes avec fractions}




\begin{exercice}[Impôts sur le revenu]
Le calcul de l'impôt $I$ pour un revenu annuel imposable $R$ (abattement des 10\,\%\ inclus) compris entre 11 198 € et 24 872 € est basé sur la relation suivante : $I =\dfrac{14}{100}R - 857$. 
Quel est le revenu annuel imposable $R$ d'un individu qui paie 1 040 € d'impôts ?
\end{exercice}



\begin{exercice}

Trouve une fraction égale à $\dfrac{4}{3}$ dont la somme du numérateur et du dénominateur est égale à 63 (tu appelleras $x$ le numérateur de la fraction recherchée). 
\end{exercice}



\begin{exercice}[Un rectangle...]

Les longueurs sont données en cm et les aires en cm$^2$.

$L$ et $\ell$ désignent respectivement la longueur et la largeur d'un rectangle. On sait que l'aire de ce rectangle est de 230,4 et que $\dfrac{L}{\ell}=\dfrac{5}{2}$.

\begin{colenumerate}{1} 
\item Calculer les mesures exactes de la longueur et de la largeur de ce rectangle.
\item Calculer la mesure exacte du périmètre de ce rectangle.
\end{colenumerate} 
\end{exercice}



\begin{exercice}[Drôle de nombre]
Si on retranche un même nombre au numérateur et au dénominateur de la fraction $\dfrac{4}{5}$, on obtient la fraction $\dfrac{5}{4}$.

Trouver ce nombre.
\end{exercice}
