\begin{enigme}[Quel âge ?]
Dialogue entre un père et son fils.

Le fils : \og  Nous avons 27 ans de différence ! \fg.

Le père : \og  Et j'ai le quadruple de l'âge que tu avais quand j'avais l'âge que tu as !! \fg.

Mais quel âge ont-ils ?
\end{enigme}

\vspace{2em}

\begin{enigme}[À la ferme]
Le fermier a compté 83 têtes et 236 pattes.

Combien cela fait-il de bipèdes et de quadrupèdes ?

Combien de poules a-t-il ?

Quel âge a le fermier ?
\end{enigme}

\vspace{2em}

\begin{enigme}[Et ainsi de suite...]
Démontre que les nombres suivants : $\dfrac{1}{2} ; \dfrac{2}{3} ; \dfrac{3}{4} ; \dfrac{4}{5} ; ... ; \dfrac{n}{n+1} ; ...$ sont rangés dans l'ordre croissant et sont tous inférieurs à 1 !
\end{enigme}






