\section{Vocabulaire}\label{EQcours01}



\begin{definition}
Une \MotDefinition{équation}{} est une expression dans laquelle il y a toujours un signe égal et une ou plusieurs inconnues (désignées chacune par une lettre, en général).
\end{definition}


\begin{exemple*1}

$2x^2 -5 = x + 10$ est une équation où l'inconnue est désignée par la lettre $x$.

Cette équation a deux membres : $2x^2 -5$ (membre de gauche) et $x + 10$ (membre de droite).
\end{exemple*1}

\vspace{2em}

\begin{definition}
\MotDefinition{Résoudre une équation}{} d'inconnue $x$, c'est déterminer toutes les valeurs de $x$ (si elles existent) pour que l'égalité soit vraie. Chacune de ces valeurs est appelée \MotDefinition{solution de l'équation}{}.
\end{definition} 
 


\begin{exemple*1}
Les solutions de l'équation $2x^2 -5 = x + 10$ sont les valeurs du nombre $x$ pour lesquelles l'égalité  $2x^2 -5 = x + 10$  est vérifiée.
\end{exemple*1}

\begin{exemple*1}
3 est-il une solution de l'équation $2x^2 -5 = x + 10$ ?

\correction

Pour $x = 3$, on calcule séparément $2x^2 -5$ et $x + 10$ :
\begin{align*}
2x^2 -5 &= 2  \times 3^2 -5 = 2  \times 9 -5 = 13 \\
x + 10 &= 3 + 10 = 13 \\
\end{align*}
 
On constate qu'il y a égalité donc 3 est une solution de l'équation $2x^2 -5 = x + 10$.

\end{exemple*1}


%%%
%%% ATTENTION NEWPAGE MISE EN PAGE
%%%
\newpage

\section{Résolution d'une équation du premier degré}\label{EQcours02}
 

\begin{propriete}
Pour tous nombres $a$, $b$ et $c$ :
\begin{itemize}
    \item Une égalité reste vraie si on ajoute ou si on soustrait un même nombre à ses deux membres.
        \subitem si $a = b$ alors $a + c = b + c$
        \subitem si $a = b$ alors $a -c = b -c$
    \item Une égalité reste vraie si on multiplie ou si on divise ses deux membres par un même nombre non nul.
        \subitem si $a = b$ alors $a  \times c = b  \times c$
        \subitem si $a = b$ alors $\dfrac{a}{c}=\dfrac{b}{c}$ (où $c \neq 0$)
\end{itemize}
\end{propriete}








\begin{exemple*1}

Résous l'équation $7x + 2 = 4x + 9$.

\correction

\vspace{.5em}

\begin{tabular}{lcl}
$7x + 2 = 4x + 9$ & \multirow{2}{*}{$\longrightarrow$} & \textbf{On élimine les termes en $x$ dans le membre de} \\
$7x + 2 \boldsymbol{-4x} = 4x + 9 \boldsymbol{-4x}$ &  & \textbf{droite} en retranchant $\boldsymbol{4x}$ aux deux membres.\\
& & \\
%
$3x + 2 = 9$ & \multirow{2}{*}{$\longrightarrow$} & \textbf{On isole le terme en $x$ dans le membre de}\\
$3x + 2 \boldsymbol{-2} = 9 \boldsymbol{-2}$ & & \textbf{gauche} en retranchant $\boldsymbol{2}$ aux deux membres.\\
& & \\
%
$3x = 7$ & \multirow{3}{*}{$\longrightarrow$} & \multirow{3}{*}{\textbf{On cherche la valeur de l'inconnue $x$} en divisant les deux membres par $\boldsymbol{3}$.}\\
$\dfrac{3x}{\boldsymbol{3}}=\dfrac{7}{\boldsymbol{3}}$ &  & \\
$x=\dfrac{7}{3}$ &  & \\
%
\end{tabular}

\vspace{1em}

Ainsi $7x + 2 = 4x + 9$ pour l'unique solution $x=\dfrac{7}{3}$. 

Puis, on vérifie que $\dfrac{7}{3}$ est une solution de l'équation $7x + 2 = 4x + 9$ en appliquant la méthode présentée dans la partie \ref{EQcours01}.
\end{exemple*1}


%%%
%%% ATTENTION NEWPAGE MISE EN PAGE
%%%
\newpage

\section{Résolution de problème} 


\begin{definition}
Mettre en équation un problème, c'est traduire son énoncé par une égalité mathématique.
\end{definition}


\begin{exemple*1}

Trouve le nombre tel que son quintuple augmenté de 7 soit égal à 3.

\correction

\vspace{.5em}

\begin{tabular}{llcl}
Étape n°1 : Choix de & \multirow{2}{*}{Soit $x$ le nombre cherché.} & \multirow{2}{*}{$\longrightarrow$} & \multirow{2}{*}{On note généralement l'inconnue $x$.} \\
l'inconnue & & & \\
& & & \\
%
Étape n°2 : Mise en & Le quintuple du nombre & \multirow{2}{*}{$\longrightarrow$} & On exprime les informations \\
équation & augmenté de 7 est $5x + 7$. &  & données dans l'énoncé en fonction de $x$. \\
 & 5x + 7 = 3 & $\longrightarrow$ & La phrase de l'énoncé se traduit ainsi.\\
 & & & \\
%
Étape n°3 : Résolution  & $5x + 7=3$ &  & \\
de l'équation           & $5x + 7 \boldsymbol{-7} = 3\boldsymbol{-7}$ &  & \\
                        & $5x=-4$ & $\longrightarrow$ & On résout l'équation à l'aide des \\
                        & $x=\dfrac{-4}{5}$ & & propriétés de la partie \ref{EQcours02}.\\
 & & & \\
%
Étape n°4 : Vérification & &  &  \\
que la valeur trouvée & $5 \times \left(-\dfrac{4}{5}\right)+7 = -4 +7=3 $ & $\longrightarrow$ & On calcule. Le quintuple de $-\dfrac{4}{5}$ \\
est solution du problème & &  & augmenté de 7 est égal à 3. \\
 & & & \\
%
Étape n°5 : Conclusion & \multicolumn{3}{l}{Le nombre cherché est donc $-\dfrac{4}{5}$.} \\
\end{tabular}
 
\end{exemple*1}